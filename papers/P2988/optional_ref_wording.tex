\documentclass[a4paper,10pt,oneside,openany,final,article]{memoir}
%% Common header for WG21 proposals ? mainly taken from C++ standard draft source
%%

%%--------------------------------------------------
%% basics
% \documentclass[a4paper,11pt,oneside,openany,final,article]{memoir}

\usepackage[american]
           {babel}        % needed for iso dates
\usepackage[iso,american]
           {isodate}      % use iso format for dates
\usepackage[final]
           {listings}     % code listings
\usepackage{longtable}    % auto-breaking tables
\usepackage{ltcaption}    % fix captions for long tables
\usepackage{relsize}      % provide relative font size changes
\usepackage{textcomp}     % provide \text{l,r}angle
\usepackage{underscore}   % remove special status of '_' in ordinary text
\usepackage{parskip}      % handle non-indented paragraphs "properly"
\usepackage{array}        % new column definitions for tables
\usepackage[normalem]{ulem}
\usepackage{enumitem}
\usepackage{color}        % define colors for strikeouts and underlines
\usepackage{amsmath}      % additional math symbols
\usepackage{mathrsfs}     % mathscr font
\usepackage[final]{microtype}
\usepackage[splitindex,original]{imakeidx}
\usepackage{multicol}
\usepackage{lmodern}
\usepackage{xcolor}
\usepackage[T1]{fontenc}
\usepackage{graphicx}
\usepackage{hyperref}
% \usepackage[pdftex,
%             bookmarks=true,
%             bookmarksnumbered=true,
%             pdfpagelabels=true,
%             pdfpagemode=UseOutlines,
%             pdfstartview=FitH,
%             linktocpage=true,
%             colorlinks=true,
%             plainpages=false,
%             allcolors={blue},
%             allbordercolors={white}
%             ]{hyperref}

% \usepackage[pdftex,
%   pdfauthor={Steve Downey},
%   pdftitle={dXXXpX},
%   pdfkeywords={},
%   pdfsubject={},
%   pdfcreator={Emacs 28.1.50 (Org mode 9.5.2)},
%   pdflang={English}]{hyperref}

\usepackage{memhfixc}     % fix interactions between hyperref and memoir
\usepackage{expl3}
\usepackage{xparse}
\usepackage{xstring}

\usepackage{url}  % urls in ref.bib
\usepackage{tabularx}  % don't use the C++ standard's fancy tables, they come with captions!

\special{pdf:minorversion 5} %set minorversion
\special{pdf:compresslevel 9} %set minorversion
\special{pdf:objcompresslevel 9} %set minorversion

% \pdfminorversion=5
% \pdfcompresslevel=9
% \pdfobjcompresslevel=2

\renewcommand\RSsmallest{5.5pt}  % smallest font size for relsize

%!TEX root = std.tex
%% layout.tex -- set overall page appearance

%%--------------------------------------------------
%%  set page size, type block size, type block position

\setlrmarginsandblock{2.245cm}{2.245cm}{*}
\setulmarginsandblock{2.5cm}{2.5cm}{*}

%%--------------------------------------------------
%%  set header and footer positions and sizes

\setheadfoot{\onelineskip}{2\onelineskip}
\setheaderspaces{*}{2\onelineskip}{*}

%%--------------------------------------------------
%%  make miscellaneous adjustments, then finish the layout
\setmarginnotes{7pt}{7pt}{0pt}
\checkandfixthelayout

%%--------------------------------------------------
%% If there is insufficient stretchable vertical space on a page,
%% TeX will not properly consider penalties for a good page break,
%% even if \raggedbottom (default for oneside, not for twoside)
%% is in effect.
\raggedbottom
\addtolength{\topskip}{0pt plus 20pt}

%%--------------------------------------------------
%% Place footnotes at the bottom of the page, rather
%% than immediately following the main text.
\feetatbottom

%%--------------------------------------------------
%% Paragraph and bullet numbering

% create a new counter that resets for each new subclause
\newcommand{\newsubclausecounter}[1]{
\newcounter{#1}
\counterwithin{#1}{chapter}
\counterwithin{#1}{section}
\counterwithin{#1}{subsection}
\counterwithin{#1}{subsubsection}
\counterwithin{#1}{paragraph}
\counterwithin{#1}{subparagraph}
}

\newsubclausecounter{Paras}
\newcounter{Bullets1}[Paras]
\newcounter{Bullets2}[Bullets1]
\newcounter{Bullets3}[Bullets2]
\newcounter{Bullets4}[Bullets3]

\makeatletter
\newcommand{\parabullnum}[2]{%
\stepcounter{#1}%
\noindent\makebox[0pt][l]{\makebox[#2][r]{%
\scriptsize\raisebox{.7ex}%
{%
\ifnum \value{Paras}>0
\ifnum \value{Bullets1}>0 (\fi%
                          \arabic{Paras}%
\ifnum \value{Bullets1}>0 .\arabic{Bullets1}%
\ifnum \value{Bullets2}>0 .\arabic{Bullets2}%
\ifnum \value{Bullets3}>0 .\arabic{Bullets3}%
\fi\fi\fi%
\ifnum \value{Bullets1}>0 )\fi%
\fi%
}%
\hspace{\@totalleftmargin}\quad%
}}}
\makeatother

% Register our intent to number the next paragraph. Don't actually number it
% yet, because we might have a paragraph break before we see its contents (for
% example, if the paragraph begins with a note or example).
\def\pnum{%
\global\def\maybeaddpnum{\global\def\maybeaddpnum{}\parabullnum{Paras}{0pt}}%
\everypar=\expandafter{\the\everypar\maybeaddpnum}%
}

% Leave more room for section numbers in TOC
\cftsetindents{section}{1.5em}{3.0em}

%!TEX root = std.tex
%% styles.tex -- set styles for:
%     chapters
%     pages
%     footnotes

%%--------------------------------------------------
%%  create chapter style

\makechapterstyle{cppstd}{%
  \renewcommand{\beforechapskip}{\onelineskip}
  \renewcommand{\afterchapskip}{\onelineskip}
  \renewcommand{\chapternamenum}{}
  \renewcommand{\chapnamefont}{\chaptitlefont}
  \renewcommand{\chapnumfont}{\chaptitlefont}
  \renewcommand{\printchapternum}{\chapnumfont\thechapter\quad}
  \renewcommand{\afterchapternum}{}
}

%%--------------------------------------------------
%%  create page styles

\makepagestyle{cpppage}
\makeevenhead{cpppage}{}
\makeoddhead{cpppage}{}
\makeevenfoot{cpppage}{\leftmark}{}{\thepage}
\makeoddfoot{cpppage}{\leftmark}{}{\thepage}

\makeatletter
\makepsmarks{cpppage}{%
  \let\@mkboth\markboth
  \def\chaptermark##1{\markboth{##1}{##1}}%
  \def\sectionmark##1{\markboth{%
    \ifnum \c@secnumdepth>\z@
      \textsection\space\thesection
    \fi
    }{\rightmark}}%
  \def\subsectionmark##1{\markboth{%
    \ifnum \c@secnumdepth>\z@
      \textsection\space\thesubsection
    \fi
    }{\rightmark}}%
  \def\subsubsectionmark##1{\markboth{%
    \ifnum \c@secnumdepth>\z@
      \textsection\space\thesubsubsection
    \fi
    }{\rightmark}}%
  \def\paragraphmark##1{\markboth{%
    \ifnum \c@secnumdepth>\z@
      \textsection\space\theparagraph
    \fi
    }{\rightmark}}}
\makeatother

\aliaspagestyle{chapter}{cpppage}

%%--------------------------------------------------
%%  set heading styles for main matter
\newcommand{\beforeskip}{-.7\onelineskip plus -1ex}
\newcommand{\afterskip}{.3\onelineskip minus .2ex}

\setbeforesecskip{\beforeskip}
\setsecindent{0pt}
\setsecheadstyle{\large\bfseries\raggedright}
\setaftersecskip{\afterskip}

\setbeforesubsecskip{\beforeskip}
\setsubsecindent{0pt}
\setsubsecheadstyle{\large\bfseries\raggedright}
\setaftersubsecskip{\afterskip}

\setbeforesubsubsecskip{\beforeskip}
\setsubsubsecindent{0pt}
\setsubsubsecheadstyle{\normalsize\bfseries\raggedright}
\setaftersubsubsecskip{\afterskip}

\setbeforeparaskip{\beforeskip}
\setparaindent{0pt}
\setparaheadstyle{\normalsize\bfseries\raggedright}
\setafterparaskip{\afterskip}

\setbeforesubparaskip{\beforeskip}
\setsubparaindent{0pt}
\setsubparaheadstyle{\normalsize\bfseries\raggedright}
\setaftersubparaskip{\afterskip}

%%--------------------------------------------------
% set heading style for annexes
\newcommand{\Annex}[3]{\chapter[#2]{(#3)\protect\\#2\hfill[#1]}\relax\annexlabel{#1}}
\newcommand{\infannex}[2]{\Annex{#1}{#2}{informative}\addxref{#1}}
\newcommand{\normannex}[2]{\Annex{#1}{#2}{normative}\addxref{#1}}

%%--------------------------------------------------
%%  set footnote style
\footmarkstyle{\smaller#1) }

%%--------------------------------------------------
% set style for main text
\setlength{\parindent}{0pt}
\setlength{\parskip}{1ex}

% set style for lists (itemizations, enumerations)
\setlength{\partopsep}{0pt}
\newlist{indenthelper}{itemize}{1}
\newlist{bnflist}{itemize}{1}
\setlist[itemize]{parsep=\parskip, partopsep=0pt, itemsep=0pt, topsep=0pt,
                  beginpenalty=10 }
\setlist[enumerate]{parsep=\parskip, partopsep=0pt, itemsep=0pt, topsep=0pt}
\setlist[indenthelper]{parsep=\parskip, partopsep=0pt, itemsep=0pt, topsep=0pt, label={}}
\setlist[bnflist]{parsep=\parskip, partopsep=0pt, itemsep=0pt, topsep=0pt, label={},
                  leftmargin=\bnfindentrest, listparindent=-\bnfindentinc, itemindent=\listparindent}

%%--------------------------------------------------
%%  set caption style
\captionstyle{\centering}

%%--------------------------------------------------
%% set global styles that get reset by \mainmatter
\newcommand{\setglobalstyles}{
  \counterwithout{footnote}{chapter}
  \counterwithout{table}{chapter}
  \counterwithout{figure}{chapter}
  \renewcommand{\chaptername}{}
  \renewcommand{\appendixname}{Annex }
}

%%--------------------------------------------------
%% change list item markers to number and em-dash

\renewcommand{\labelitemi}{---\parabullnum{Bullets1}{\labelsep}}
\renewcommand{\labelitemii}{---\parabullnum{Bullets2}{\labelsep}}
\renewcommand{\labelitemiii}{---\parabullnum{Bullets3}{\labelsep}}
\renewcommand{\labelitemiv}{---\parabullnum{Bullets4}{\labelsep}}

%%--------------------------------------------------
%% set section numbering limit, toc limit
\maxsecnumdepth{subparagraph}
\setcounter{tocdepth}{1}

%%--------------------------------------------------
%% override some functions from the listings package to avoid bad page breaks
%% (copied verbatim from listings.sty version 1.6 except where commented)
\makeatletter

\def\lst@Init#1{%
    \begingroup
    \ifx\lst@float\relax\else
        \edef\@tempa{\noexpand\lst@beginfloat{lstlisting}[\lst@float]}%
        \expandafter\@tempa
    \fi
    \ifx\lst@multicols\@empty\else
        \edef\lst@next{\noexpand\multicols{\lst@multicols}}
        \expandafter\lst@next
    \fi
    \ifhmode\ifinner \lst@boxtrue \fi\fi
    \lst@ifbox
        \lsthk@BoxUnsafe
        \hbox to\z@\bgroup
             $\if t\lst@boxpos \vtop
        \else \if b\lst@boxpos \vbox
        \else \vcenter \fi\fi
        \bgroup \par\noindent
    \else
        \lst@ifdisplaystyle
            \lst@EveryDisplay
            % make penalty configurable
            \par\lst@beginpenalty
            \vspace\lst@aboveskip
        \fi
    \fi
    \normalbaselines
    \abovecaptionskip\lst@abovecaption\relax
    \belowcaptionskip\lst@belowcaption\relax
    \lst@MakeCaption t%
    \lsthk@PreInit \lsthk@Init
    \lst@ifdisplaystyle
        \global\let\lst@ltxlabel\@empty
        \if@inlabel
            \lst@ifresetmargins
                \leavevmode
            \else
                \xdef\lst@ltxlabel{\the\everypar}%
                \lst@AddTo\lst@ltxlabel{%
                    \global\let\lst@ltxlabel\@empty
                    \everypar{\lsthk@EveryLine\lsthk@EveryPar}}%
            \fi
        \fi
        % A section heading might have set \everypar to apply a \clubpenalty
        % to the following paragraph, changing \everypar in the process.
        % Unconditionally overriding \everypar is a bad idea.
        % \everypar\expandafter{\lst@ltxlabel
        %                      \lsthk@EveryLine\lsthk@EveryPar}%
    \else
        \everypar{}\let\lst@NewLine\@empty
    \fi
    \lsthk@InitVars \lsthk@InitVarsBOL
    \lst@Let{13}\lst@MProcessListing
    \let\lst@Backslash#1%
    \lst@EnterMode{\lst@Pmode}{\lst@SelectCharTable}%
    \lst@InitFinalize}

\def\lst@DeInit{%
    \lst@XPrintToken \lst@EOLUpdate
    \global\advance\lst@newlines\m@ne
    \lst@ifshowlines
        \lst@DoNewLines
    \else
        \setbox\@tempboxa\vbox{\lst@DoNewLines}%
    \fi
    \lst@ifdisplaystyle \par\removelastskip \fi
    \lsthk@ExitVars\everypar{}\lsthk@DeInit\normalbaselines\normalcolor
    \lst@MakeCaption b%
    \lst@ifbox
        \egroup $\hss \egroup
        \vrule\@width\lst@maxwidth\@height\z@\@depth\z@
    \else
        \lst@ifdisplaystyle
            % make penalty configurable
            \par\lst@endpenalty
            \vspace\lst@belowskip
        \fi
    \fi
    \ifx\lst@multicols\@empty\else
        \def\lst@next{\global\let\@checkend\@gobble
                      \endmulticols
                      \global\let\@checkend\lst@@checkend}
        \expandafter\lst@next
    \fi
    \ifx\lst@float\relax\else
        \expandafter\lst@endfloat
    \fi
    \endgroup}


\def\lst@NewLine{%
    \ifx\lst@OutputBox\@gobble\else
        \par
        % add configurable penalties
        \lst@ifeolsemicolon
          \lst@semicolonpenalty
          \lst@eolsemicolonfalse
        \else
          \lst@domidpenalty
        \fi
        % Manually apply EveryLine and EveryPar; do not depend on \everypar
        \noindent \hbox{}\lsthk@EveryLine%
        % \lsthk@EveryPar uses \refstepcounter which balloons the PDF
    \fi
    \global\advance\lst@newlines\m@ne
    \lst@newlinetrue}

% new macro for empty lines, avoiding an \hbox that cannot be discarded
\def\lst@DoEmptyLine{%
  \ifvmode\else\par\fi\lst@emptylinepenalty
  \vskip\parskip
  \vskip\baselineskip
  % \lsthk@EveryLine has \lst@parshape, i.e., \parshape, which causes an \hbox
  % \lsthk@EveryPar increments line counters; \refstepcounter balloons the PDF
  \global\advance\lst@newlines\m@ne
  \lst@newlinetrue}

\def\lst@DoNewLines{
    \@whilenum\lst@newlines>\lst@maxempty \do
        {\lst@ifpreservenumber
            \lsthk@OnEmptyLine
            \global\advance\c@lstnumber\lst@advancelstnum
         \fi
         \global\advance\lst@newlines\m@ne}%
    \@whilenum \lst@newlines>\@ne \do
        % special-case empty printing of lines
        {\lsthk@OnEmptyLine\lst@DoEmptyLine}%
    \ifnum\lst@newlines>\z@ \lst@NewLine \fi}

% add keys for configuring before/end vertical penalties
\lst@Key{beginpenalty}\relax{\def\lst@beginpenalty{\penalty #1}}
\let\lst@beginpenalty\@empty
\lst@Key{midpenalty}\relax{\def\lst@midpenalty{\penalty #1}}
\let\lst@midpenalty\@empty
\lst@Key{endpenalty}\relax{\def\lst@endpenalty{\penalty #1}}
\let\lst@endpenalty\@empty
\lst@Key{emptylinepenalty}\relax{\def\lst@emptylinepenalty{\penalty #1}}
\let\lst@emptylinepenalty\@empty
\lst@Key{semicolonpenalty}\relax{\def\lst@semicolonpenalty{\penalty #1}}
\let\lst@semicolonpenalty\@empty

\lst@AddToHook{InitVars}{\let\lst@domidpenalty\@empty}
\lst@AddToHook{InitVarsEOL}{\let\lst@domidpenalty\lst@midpenalty}

% handle semicolons and closing braces (could be in \lstdefinelanguage as well)
\def\lst@eolsemicolontrue{\global\let\lst@ifeolsemicolon\iftrue}
\def\lst@eolsemicolonfalse{\global\let\lst@ifeolsemicolon\iffalse}
\lst@AddToHook{InitVars}{
  \global\let\lst@eolsemicolonpending\@empty
  \lst@eolsemicolonfalse
}
% If we found a semicolon or closing brace while parsing the current line,
% inform the subsequent \lst@NewLine about it for penalties.
\lst@AddToHook{InitVarsEOL}{%
  \ifx\lst@eolsemicolonpending\relax
    \lst@eolsemicolontrue
    \global\let\lst@eolsemicolonpending\@empty
  \fi%
}
\lst@AddToHook{SelectCharTable}{%
  % In theory, we should only detect trailing semicolons or braces,
  % but that would require un-doing the marking for any other character.
  % The next best thing is to undo the marking for closing parentheses,
  % because loops or if statements are the only places where we will
  % reasonably have a semicolon in the middle of a line, and those all
  % end with a closing parenthesis.
  \lst@DefSaveDef{41}\lstsaved@closeparen{%    handle closing parenthesis
    \lstsaved@closeparen
    \ifnum\lst@mode=\lst@Pmode    % regular processing mode (not a comment)
      \global\let\lst@eolsemicolonpending\@empty  % undo semicolon setting
    \fi%
  }%
  \lst@DefSaveDef{59}\lstsaved@semicolon{%     handle semicolon
    \lstsaved@semicolon
    \ifnum\lst@mode=\lst@Pmode    % regular processing mode (not a comment)
      \global\let\lst@eolsemicolonpending\relax
    \fi%
  }%
  \lst@DefSaveDef{125}\lstsaved@closebrace{%   handle closing brace
    \lst@eolsemicolonfalse        % do not break before a closing brace
    \lstsaved@closebrace          % might invoke \lst@NewLine
    \ifnum\lst@mode=\lst@Pmode    % regular processing mode (not a comment)
      \global\let\lst@eolsemicolonpending\relax
    \fi%
  }%
}

\makeatother

%!TEX root = std.tex

% Definitions and redefinitions of special commands
%
% For macros that are likely to appear inside code blocks, we may provide a
% "macro length" correction value, which can be used to determine the effective
% column number where an emitted character appears, by adding the macro length,
% plus 2 for the escape-'@'s, plus 2 for the braces, to the real column number.
%
% For example:
%
% \newcommand{\foo[1]{#1}  % macro length: 4
%
% \begin{codeblock}
% int a = 10;         // comment on column 20
% int b = @\foo{x}@;          // comment also on effective column 20
%
% Here the real column of the second comment start is offset by 8 (4 + macro length).

%%--------------------------------------------------
%% Difference markups
%%--------------------------------------------------
\definecolor{addclr}{rgb}{0,.6,.6}
\definecolor{remclr}{rgb}{1,0,0}
\definecolor{noteclr}{rgb}{0,0,1}

\renewcommand{\added}[1]{\textcolor{addclr}{\uline{#1}}}
\newcommand{\removed}[1]{\textcolor{remclr}{\sout{#1}}}
\renewcommand{\changed}[2]{\removed{#1}\added{#2}}

\newcommand{\nbc}[1]{[#1]\ }
\newcommand{\addednb}[2]{\added{\nbc{#1}#2}}
\newcommand{\removednb}[2]{\removed{\nbc{#1}#2}}
\newcommand{\changednb}[3]{\removednb{#1}{#2}\added{#3}}
\newcommand{\remitem}[1]{\item\removed{#1}}

\newcommand{\ednote}[1]{\textcolor{noteclr}{[Editor's note: #1] }}

\newenvironment{addedblock}{\color{addclr}}{\color{black}}
\newenvironment{removedblock}{\color{remclr}}{\color{black}}

%%--------------------------------------------------
%% Grammar extraction.
%%--------------------------------------------------
\def\gramSec[#1]#2{}

\makeatletter
\newcommand{\FlushAndPrintGrammar}{%
\immediate\closeout\XTR@out%
\immediate\openout\XTR@out=std-gram-dummy.tmp%
\def\gramSec[##1]##2{\rSec1[##1]{##2}}%
\input{std-gram.ext}%
}
\makeatother

%%--------------------------------------------------
% Escaping for index entries. Replaces ! with "! throughout its argument.
%%--------------------------------------------------
\def\indexescape#1{\doindexescape#1\stopindexescape!\doneindexescape}
\def\doindexescape#1!{#1"!\doindexescape}
\def\stopindexescape#1\doneindexescape{}

%%--------------------------------------------------
%% Cross-references.
%%--------------------------------------------------
\newcommand{\addxref}[1]{%
 \glossary[xrefindex]{\indexescape{#1}}{(\ref{\indexescape{#1}})}%
}

%%--------------------------------------------------
%% Sectioning macros.
% Each section has a depth, an automatically generated section
% number, a name, and a short tag.  The depth is an integer in
% the range [0,5].  (If it proves necessary, it wouldn't take much
% programming to raise the limit from 5 to something larger.)
%%--------------------------------------------------

% Set the xref label for a clause to be "Clause n", not just "n".
\makeatletter
\newcommand{\customlabel}[2]{%
\@bsphack \begingroup \protected@edef \@currentlabel {\protect \M@TitleReference{#2}{\M@currentTitle}}\MNR@label{#1}\endgroup \@esphack%
}
\makeatother
\newcommand{\clauselabel}[1]{\customlabel{#1}{Clause \thechapter}}
\newcommand{\annexlabel}[1]{\customlabel{#1}{Annex \thechapter}}

% Use prefix "Annex" in the table of contents
\newcommand{\annexnumberlinebox}[2]{Annex #2\space}

% The basic sectioning command.  Example:
%    \Sec1[intro.scope]{Scope}
% defines a first-level section whose name is "Scope" and whose short
% tag is intro.scope.  The square brackets are mandatory.
\def\Sec#1[#2]#3{%
\ifcase#1\let\s=\chapter\let\l=\clauselabel
      \or\let\s=\section\let\l=\label
      \or\let\s=\subsection\let\l=\label
      \or\let\s=\subsubsection\let\l=\label
      \or\let\s=\paragraph\let\l=\label
      \or\let\s=\subparagraph\let\l=\label
      \fi%
\s[#3]{#3\hfill[#2]}\l{#2}\addxref{#2}%
}

% A convenience feature (mostly for the convenience of the Project
% Editor, to make it easy to move around large blocks of text):
% the \rSec macro is just like the \Sec macro, except that depths
% relative to a global variable, SectionDepthBase.  So, for example,
% if SectionDepthBase is 1,
%   \rSec1[temp.arg.type]{Template type arguments}
% is equivalent to
%   \Sec2[temp.arg.type]{Template type arguments}
\newcounter{SectionDepthBase}
\newcounter{SectionDepth}

\def\rSec#1[#2]#3{%
\setcounter{SectionDepth}{#1}
\addtocounter{SectionDepth}{\value{SectionDepthBase}}
\Sec{\arabic{SectionDepth}}[#2]{#3}}

%%--------------------------------------------------
% Indexing
%%--------------------------------------------------

% Layout of general index
\newcommand{\rSecindex}[2]{\section*{#2}\pdfbookmark[1]{#2}{pdf.idx.#1.#2}\label{idx.#1.#2}}

% locations
\newcommand{\indextext}[1]{\index[generalindex]{#1}}
\newcommand{\indexlibrary}[1]{\index[libraryindex]{#1}}
\newcommand{\indexhdr}[1]{\index[headerindex]{\idxhdr{#1}}}
\newcommand{\indexconcept}[1]{\index[conceptindex]{#1}}
\newcommand{\indexgram}[1]{\index[grammarindex]{#1}}

% Collation helper: When building an index key, replace all macro definitions
% in the key argument with a no-op for purposes of collation.
\newcommand{\nocode}[1]{#1}
\newcommand{\idxmname}[1]{\_\_#1\_\_}
\newcommand{\idxCpp}{C++}

% \indeximpldef synthesizes a collation key from the argument; that is, an
% invocation \indeximpldef{arg} emits an index entry `key@arg`, where `key`
% is derived from `arg` by replacing the folowing list of commands with their
% bare content. This allows, say, collating plain text and code.
\newcommand{\indeximpldef}[1]{%
\let\otextup\textup%
\let\textup\nocode%
\let\otcode\tcode%
\let\tcode\nocode%
\let\oexposid\exposid%
\let\exposid\nocode%
\let\ogrammarterm\grammarterm%
\let\grammarterm\nocode%
\let\omname\mname%
\let\mname\idxmname%
\let\oCpp\Cpp%
\let\Cpp\idxCpp%
\let\oBreakableUnderscore\BreakableUnderscore%  See the "underscore" package.
\let\BreakableUnderscore\textunderscore%
\edef\x{#1}%
\let\tcode\otcode%
\let\exposid\oexposid%
\let\grammarterm\gterm%
\let\mname\omname%
\let\Cpp\oCpp%
\let\BreakableUnderscore\oBreakableUnderscore%
\index[impldefindex]{\x@#1}%
\let\grammarterm\ogrammarterm%
\let\textup\otextup%
}

\newcommand{\indexdefn}[1]{\indextext{#1}}
\newcommand{\idxbfpage}[1]{\textbf{\hyperpage{#1}}}
\newcommand{\indexgrammar}[1]{\indextext{#1}\indexgram{#1|idxbfpage}}
% This command uses the "cooked" \indeximpldef command to emit index
% entries; thus they only work for simple index entries that do not contain
% special indexing instructions.
\newcommand{\impldef}[1]{\indeximpldef{#1}imple\-men\-ta\-tion-defined}
% \impldefplain passes the argument directly to the index, allowing you to
% use special indexing instructions (!, @, |).
\newcommand{\impldefplain}[1]{\index[impldefindex]{#1}implementation-defined}

% appearance
% avoid \tcode to avoid falling victim of \tcode redefinition in CodeBlockSetup
\newcommand{\idxcode}[1]{#1@\CodeStylex{#1}}
\newcommand{\idxconcept}[1]{#1@\CodeStylex{#1}}
\newcommand{\idxexposconcept}[1]{#1@\CodeStylex{\placeholder{#1}}}
\newcommand{\idxhdr}[1]{#1@\CodeStylex{<#1>}}
\newcommand{\idxgram}[1]{#1@\gterm{#1}}
\newcommand{\idxterm}[1]{#1@\term{#1}}
\newcommand{\idxxname}[1]{__#1@\xname{#1}}

% library index entries
\newcommand{\indexlibraryglobal}[1]{\indexlibrary{\idxcode{#1}}}
\newcommand{\indexlibrarymisc}[2]{\indexlibrary{#1!#2}}
\newcommand{\indexlibraryctor}  [1]{\indexlibrarymisc{\idxcode{#1}}{constructor}}
\newcommand{\indexlibrarydtor}  [1]{\indexlibrarymisc{\idxcode{#1}}{destructor}}
\newcommand{\indexlibraryzombie}[1]{\indexlibrarymisc{\idxcode{#1}}{zombie}}

% class member library index
\newcommand{\indexlibraryboth}[2]{\indexlibrarymisc{#1}{#2}\indexlibrarymisc{#2}{#1}}
\newcommand{\indexlibrarymember}[2]{\indexlibraryboth{\idxcode{#1}}{\idxcode{#2}}}
\newcommand{\indexlibrarymemberexpos}[2]{\indexlibraryboth{\idxcode{#1}}{#2@\exposid{#2}}}
\newcommand{\indexlibrarymemberx}[2]{\indexlibrarymisc{\idxcode{#1}}{\idxcode{#2}}}

\newcommand{\libglobal}[1]{\indexlibraryglobal{#1}#1}
\newcommand{\libmember}[2]{\indexlibrarymember{#1}{#2}#1}
\newcommand{\libspec}[2]{\indexlibrarymemberx{#1}{#2}#1}

% index for library headers
\newcommand{\libheader}[1]{\indexhdr{#1}\tcode{<#1>}}
\newcommand{\indexheader}[1]{\index[headerindex]{\idxhdr{#1}|idxbfpage}}
\newcommand{\libheaderdef}[1]{\indexheader{#1}\tcode{<#1>}}
\newcommand{\libnoheader}[1]{\indextext{\idxhdr{#1}!absence thereof}\tcode{<#1>}}
\newcommand{\libheaderrefx}[2]{\libheader{#1}\iref{#2}}
\newcommand{\libheaderref}[1]{\libheaderrefx{#1}{#1.syn}}
\newcommand{\libdeprheaderref}[1]{\libheaderrefx{#1}{depr.#1.syn}}

%%--------------------------------------------------
% General code style
%%--------------------------------------------------
\newcommand{\CodeStyle}{\ttfamily}
\newcommand{\CodeStylex}[1]{\texttt{\protect\frenchspacing #1}}

\definecolor{grammar-gray}{gray}{0.2}

% General grammar style
\newcommand{\GrammarStylex}[1]{\textcolor{grammar-gray}{\textsf{\textit{#1}}}}

% Code and definitions embedded in text.
\newcommand{\tcode}[1]{\CodeStylex{#1}}
\newcommand{\term}[1]{\textit{#1}}
\newcommand{\gterm}[1]{\GrammarStylex{#1}}
\newcommand{\fakegrammarterm}[1]{\gterm{#1}}
\newcommand{\keyword}[1]{\texorpdfstring{\tcode{#1}\protect\indextext{\idxcode{#1}!keyword}}{#1}}                % macro length: 8
\newcommand{\grammarterm}[1]{\texorpdfstring{\protect\indexgram{\idxgram{#1}}\gterm{#1}}{#1}}
\newcommand{\grammartermnc}[1]{\indexgram{\idxgram{#1}}\gterm{#1\nocorr}}
\newcommand{\regrammarterm}[1]{\textit{#1}}
\newcommand{\placeholder}[1]{\textit{#1}}                                   % macro length: 12
\newcommand{\placeholdernc}[1]{\textit{#1\nocorr}}                          % macro length: 14
\newcommand{\exposid}[1]{\tcode{\placeholder{#1}}}                          % macro length: 8
\newcommand{\exposidnc}[1]{\tcode{\placeholdernc{#1}}\itcorr[-1]}           % macro length: 10
\newcommand{\defnxname}[1]{\indextext{\idxxname{#1}}\xname{#1}}
\newcommand{\defnlibxname}[1]{\indexlibrary{\idxxname{#1}}\xname{#1}}

% Non-compound defined term.
\newcommand{\defn}[1]{\defnx{#1}{#1}}
% Defined term with different index entry.
\newcommand{\defnx}[2]{\indexdefn{#2}\textit{#1}}
% Compound defined term with 'see' for primary term.
% Usage: \defnadj{trivial}{class}
\newcommand{\defnadj}[2]{\indextext{#1 #2|see{#2, #1}}\indexdefn{#2!#1}\textit{#1 #2}}
% Compound defined term with a different form for the primary noun.
% Usage: \defnadjx{scalar}{types}{type}
\newcommand{\defnadjx}[3]{\indextext{#1 #3|see{#3, #1}}\indexdefn{#3!#1}\textit{#1 #2}}

% Macros used for the grammar of std::format format specifications.
% FIXME: For now, keep the format grammar productions out of the index, since
% they conflict with the main grammar.
% Consider renaming these en masse (to fmt-* ?) to avoid this problem.
\newcommand{\fmtnontermdef}[1]{{\BnfNontermshape#1\itcorr}\textnormal{:}}
\newcommand{\fmtgrammarterm}[1]{\gterm{#1}}

%%--------------------------------------------------
%% allow line break if needed for justification
%%--------------------------------------------------
\newcommand{\brk}{\discretionary{}{}{}}

%%--------------------------------------------------
%% Macros for funky text
%%--------------------------------------------------
\newcommand{\Cpp}{\texorpdfstring{C\kern-0.05em\protect\raisebox{.35ex}{\textsmaller[2]{+\kern-0.05em+}}}{C++}}
\newcommand{\CppIII}{\Cpp{} 2003}
\newcommand{\CppXI}{\Cpp{} 2011}
\newcommand{\CppXIV}{\Cpp{} 2014}
\newcommand{\CppXVII}{\Cpp{} 2017}
\newcommand{\CppXX}{\Cpp{} 2020}
\newcommand{\CppXXIII}{\Cpp{} 2023}
\newcommand{\CppXXVI}{\Cpp{} 2026}
\newcommand{\opt}[1]{#1\ensuremath{_\mathit{\color{black}opt}}}
\newcommand{\bigoh}[1]{\ensuremath{\mathscr{O}(#1)}}

% Make all tildes a little larger to avoid visual similarity with hyphens.
\renewcommand{\~}{\textasciitilde}
\let\OldTextAsciiTilde\textasciitilde
\renewcommand{\textasciitilde}{\protect\raisebox{-0.17ex}{\larger\OldTextAsciiTilde}}
\newcommand{\caret}{\char`\^}

%%--------------------------------------------------
%% States and operators
%%--------------------------------------------------
\newcommand{\state}[2]{\tcode{#1}\ensuremath{_{#2}}}
\newcommand{\bitand}{\ensuremath{\mathbin{\mathsf{bitand}}}}
\newcommand{\bitor}{\ensuremath{\mathbin{\mathsf{bitor}}}}
\newcommand{\xor}{\ensuremath{\mathbin{\mathsf{xor}}}}
\newcommand{\rightshift}{\ensuremath{\mathbin{\mathsf{rshift}}}}
\newcommand{\leftshift}[1]{\ensuremath{\mathbin{\mathsf{lshift}_{#1}}}}

%% Notes and examples
\newcommand{\noteintro}[1]{[\textit{#1}:}
\newcommand{\noteoutro}[1]{\textit{\,---\,#1}\kern.5pt]}

% \newnoteenvironment{ENVIRON}{BEGIN TEXT}{END TEXT}
% Creates a note-like environment beginning with BEGIN TEXT and
% ending with END TEXT. A counter with name ENVIRON indicates the
% number of this kind of note / example that has occurred in this
% subclause.
% Use tailENVIRON to avoid inserting a \par at the end.
\newcommand{\newnoteenvironment}[3]{
\newsubclausecounter{#1}
\newenvironment{tail#1}
{\par\small\stepcounter{#1}\noteintro{#2}}
{\noteoutro{#3}}
\newenvironment{#1}
{\begin{tail#1}}
% \small\par is for C++20 post-DIS compatibility
{\end{tail#1}\small\par\penalty -200}

}

\newnoteenvironment{note}{Note \arabic{note}}{end note}
\newnoteenvironment{example}{Example \arabic{example}}{end example}

\makeatletter
\let\footnote\@undefined
\global\newsavebox{\@tempfootboxa}   % must be global, to escape tables/figures
\newsavebox{\@templfootbox}
\newenvironment{footnote}{%
  \unskip\footnotemark%    no space before the mark
  \normalfont%
  \footnotesize%           text size for rendering the footnote text
  \begin{lrbox}{\@templfootbox}%       temporarily save to local box
}{%
  \end{lrbox}%
  \global\setbox\@tempfootboxa\hbox{\unhbox\@templfootbox}%  copy to global box
  \footnotetext{\unhbox\@tempfootboxa}%
}
\makeatother

%% Library function descriptions
\newcommand{\Fundescx}[1]{\textit{#1}}
\newcommand{\Fundesc}[1]{\Fundescx{#1}:\space}
\newcommand{\recommended}{\Fundesc{Recommended practice}}
\newcommand{\required}{\Fundesc{Required behavior}}
\newcommand{\constraints}{\Fundesc{Constraints}}
\newcommand{\mandates}{\Fundesc{Mandates}}
\newcommand{\expects}{\Fundesc{Preconditions}}
\newcommand{\effects}{\Fundesc{Effects}}
\newcommand{\ensures}{\Fundesc{Postconditions}}
\newcommand{\returns}{\Fundesc{Returns}}
\newcommand{\throws}{\Fundesc{Throws}}
\newcommand{\default}{\Fundesc{Default behavior}}
\newcommand{\complexity}{\Fundesc{Complexity}}
\newcommand{\remarks}{\Fundesc{Remarks}}
\newcommand{\errors}{\Fundesc{Error conditions}}
\newcommand{\sync}{\Fundesc{Synchronization}}
\newcommand{\implimits}{\Fundesc{Implementation limits}}
\newcommand{\replaceable}{\Fundesc{Replaceable}}
\newcommand{\result}{\Fundesc{Result}}
\newcommand{\returntype}{\Fundesc{Return type}}
\newcommand{\ctype}{\Fundesc{Type}}
\newcommand{\templalias}{\Fundesc{Alias template}}

%% Cross-reference
\newcommand{\xref}{\textsc{See also:}\space}
\newcommand{\xrefc}[1]{\xref{} ISO C #1}

%% Inline comma-separated parenthesized references
\ExplSyntaxOn
\NewDocumentCommand \iref { m } {
  \clist_set:Nx \l_tmpa_clist { #1 }
  \nolinebreak[3] ~ (
  \clist_map_inline:Nn \l_tmpa_clist {
    \clist_put_right:Nn \g_tmpa_clist { \ref{##1} }
  }
  \clist_use:Nn \g_tmpa_clist { ,~ }
  )
  \clist_clear:N \g_tmpa_clist
}
\ExplSyntaxOff

%% Inline non-parenthesized table reference (override memoir's \tref)
\renewcommand{\tref}[1]{\hyperref[tab:#1]{\tablerefname \nolinebreak[3] \ref*{tab:#1}}}
%% Inline non-parenthesized figure reference (override memoir's \fref)
\renewcommand{\fref}[1]{\hyperref[fig:#1]{\figurerefname \nolinebreak[3] \ref*{fig:#1}}}

%% NTBS, etc.
\verbtocs{\StrTextsmaller}|\textsmaller[1]{|
\verbtocs{\StrTextsc}|\textsc{|
\verbtocs{\StrClosingbrace}|}|
\newcommand{\ucode}[1]{%
  \textsc{u}%
  \textsmaller[2]{\kern-0.05em\protect\raisebox{.25ex}{+}}%
  \begingroup%
  \def\temp{#1}%
  \StrSubstitute{\temp}{0}{X0Z}[\temp]%
  \StrSubstitute{\temp}{1}{X1Z}[\temp]%
  \StrSubstitute{\temp}{2}{X2Z}[\temp]%
  \StrSubstitute{\temp}{3}{X3Z}[\temp]%
  \StrSubstitute{\temp}{4}{X4Z}[\temp]%
  \StrSubstitute{\temp}{5}{X5Z}[\temp]%
  \StrSubstitute{\temp}{6}{X6Z}[\temp]%
  \StrSubstitute{\temp}{7}{X7Z}[\temp]%
  \StrSubstitute{\temp}{8}{X8Z}[\temp]%
  \StrSubstitute{\temp}{9}{X9Z}[\temp]%
  \StrSubstitute{\temp}{a}{YaZ}[\temp]%
  \StrSubstitute{\temp}{b}{YbZ}[\temp]%
  \StrSubstitute{\temp}{c}{YcZ}[\temp]%
  \StrSubstitute{\temp}{d}{YdZ}[\temp]%
  \StrSubstitute{\temp}{e}{YeZ}[\temp]%
  \StrSubstitute{\temp}{f}{YfZ}[\temp]%
  \StrSubstitute{\temp}{X}{\StrTextsmaller}[\temp]%
  \StrSubstitute{\temp}{Y}{\StrTextsc}[\temp]%
  \StrSubstitute{\temp}{Z}{\StrClosingbrace}[\temp]%
  \tokenize\temp{\temp}%
  \temp%
  \endgroup%
}
\newcommand{\uname}[1]{\textsc{#1}}
\newcommand{\unicode}[2]{\ucode{#1} \uname{#2}}
\newcommand{\UAX}[1]{\texorpdfstring{UAX~\textsmaller[1]{\raisebox{0.35ex}{\#}}#1}{UAX \##1}}
\newcommand{\NTS}[1]{\textsc{#1}}
\newcommand{\ntbs}{\NTS{ntbs}}
\newcommand{\ntmbs}{\NTS{ntmbs}}
% The following are currently unused:
% \newcommand{\ntwcs}{\NTS{ntwcs}}
% \newcommand{\ntcxvis}{\NTS{ntc16s}}
% \newcommand{\ntcxxxiis}{\NTS{ntc32s}}

%% Code annotations
\newcommand{\EXPO}[1]{\textit{#1}}
\newcommand{\expos}{\EXPO{exposition only}}
\newcommand{\impdef}{\EXPO{implementation-defined}}
\newcommand{\impdefnc}{\EXPO{implementation-defined\nocorr}}
\newcommand{\impdefx}[1]{\indeximpldef{#1}\EXPO{implementation-defined}}
\newcommand{\notdef}{\EXPO{not defined}}

\newcommand{\UNSP}[1]{\textit{\texttt{#1}}}
\newcommand{\UNSPnc}[1]{\textit{\texttt{#1}\nocorr}}
\newcommand{\unspec}{\UNSP{unspecified}}
\newcommand{\unspecnc}{\UNSPnc{unspecified}}
\newcommand{\unspecbool}{\UNSP{unspecified-bool-type}}
\newcommand{\seebelow}{\UNSP{see below}}      % macro length: 0
\newcommand{\seebelownc}{\UNSPnc{see below}}  % macro length: 2
\newcommand{\unspecuniqtype}{\UNSP{unspecified unique type}}
\newcommand{\unspecalloctype}{\UNSP{unspecified allocator type}}

%% Manual insertion of italic corrections, for aligning in the presence
%% of the above annotations.
\newlength{\itcorrwidth}
\newlength{\itletterwidth}
\newcommand{\itcorr}[1][]{%
 \settowidth{\itcorrwidth}{\textit{x\/}}%
 \settowidth{\itletterwidth}{\textit{x\nocorr}}%
 \addtolength{\itcorrwidth}{-1\itletterwidth}%
 \makebox[#1\itcorrwidth]{}%
}

%% Double underscore
\newcommand{\ungap}{\kern.5pt}
\newcommand{\unun}{\textunderscore\ungap\textunderscore}
\newcommand{\xname}[1]{\tcode{\unun\ungap#1}}
\newcommand{\mname}[1]{\tcode{\unun\ungap#1\ungap\unun}}

%% An elided code fragment, /* ... */, that is formatted as code.
%% (By default, listings typeset comments as body text.)
%% Produces 9 output characters.
\newcommand{\commentellip}{\tcode{/* ...\ */}}

%% Concepts
\newcommand{\oldconceptname}[1]{Cpp17#1}
\newcommand{\oldconcept}[1]{\textit{\oldconceptname{#1}}}
\newcommand{\defnoldconcept}[1]{\indexdefn{\idxoldconcept{#1}}\oldconcept{#1}}
\newcommand{\idxoldconcept}[1]{\oldconceptname{#1}@\oldconcept{#1}}
% FIXME: A "new" oldconcept (added after C++17),
% which doesn't get a Cpp17 prefix.
\newcommand{\newoldconcept}[1]{\textit{#1}}
\newcommand{\defnnewoldconcept}[1]{\indexdefn{\idxnewoldconcept{#1}}\newoldconcept{#1}}
\newcommand{\idxnewoldconcept}[1]{#1@\newoldconcept{#1}}

\newcommand{\cname}[1]{\tcode{#1}}
\newcommand{\ecname}[1]{\tcode{\placeholder{#1}}}
\newcommand{\libconceptx}[2]{\cname{#1}\indexconcept{\idxconcept{#2}}}
\newcommand{\libconcept}[1]{\libconceptx{#1}{#1}}
\newcommand{\deflibconcept}[1]{\cname{#1}\indexlibrary{\idxconcept{#1}}\indexconcept{\idxconcept{#1}|idxbfpage}}
\newcommand{\exposconcept}[1]{\ecname{#1}\indexconcept{\idxexposconcept{#1}}}
\newcommand{\exposconceptx}[2]{\ecname{#1}\indexconcept{\idxexposconcept{#2}}}
\newcommand{\exposconceptnc}[1]{\indexconcept{\idxexposconcept{#1}}\ecname{#1}\itcorr[-1]}               % macro length: 15
\newcommand{\defexposconcept}[1]{\ecname{#1}\indexconcept{\idxexposconcept{#1}|idxbfpage}}               % macro length: 16
\newcommand{\defexposconceptnc}[1]{\ecname{#1}\indexconcept{\idxexposconcept{#1}|idxbfpage}\itcorr[-1]}  % macro length: 18

%% Ranges
\newcommand{\Range}[4]{\ensuremath{#1}\tcode{#3}\ensuremath{,}\,\penalty2000{}\tcode{#4}\ensuremath{#2}}
\newcommand{\crange}[2]{\Range{[}{]}{#1}{#2}}
\newcommand{\brange}[2]{\Range{(}{]}{#1}{#2}}
\newcommand{\orange}[2]{\Range{(}{)}{#1}{#2}}
\newcommand{\range}[2]{\Range{[}{)}{#1}{#2}}
\newcommand{\countedrange}[2]{$\tcode{#1} + \range{0}{#2}$}

%% Change descriptions
\newcommand{\diffhead}[1]{\textbf{#1:}\space}
\newcommand{\diffdef}[1]{\ifvmode\else\hfill\break\fi\diffhead{#1}}
\ExplSyntaxOn
\NewDocumentCommand \diffref { m } {
  \clist_set:Nx \l_tmpa_clist { #1 }
  \pnum
  \int_compare:nTF { \clist_count:N \l_tmpa_clist < 2 } {
    \textbf{Affected~subclause:} ~
  } {
    \textbf{Affected~subclauses:} ~
  }
  \clist_map_inline:Nn \l_tmpa_clist {
    \clist_put_right:Nn \g_tmpa_clist { \ref{##1} }
  }
  \clist_use:Nnnn \g_tmpa_clist { ~and~ } { ,~ } { ,~and~ }
  \clist_clear:N \g_tmpa_clist
}
\cs_set_eq:NN \diffrefs \diffref
\ExplSyntaxOff
% \nodiffref swallows a following \change and removes the preceding line break.
\def\nodiffref\change{\pnum
\diffhead{Change}}
\newcommand{\change}{\diffdef{Change}}
\newcommand{\rationale}{\diffdef{Rationale}}
\newcommand{\effect}{\diffdef{Effect on original feature}}
\newcommand{\effectafteritemize}{\diffhead{Effect on original feature}}
\newcommand{\difficulty}{\diffdef{Difficulty of converting}}
\newcommand{\howwide}{\diffdef{How widely used}}

%% Miscellaneous
\newcommand{\stage}[1]{\item[Stage #1:]}
\newcommand{\doccite}[1]{\textit{#1}}
\newcommand{\cvqual}[1]{\textit{#1}}
\newcommand{\cv}{\ifmmode\mathit{cv}\else\cvqual{cv}\fi}
\newcommand{\numconst}[1]{\textsl{#1}}
\newcommand{\logop}[1]{\textsc{#1}}

%%--------------------------------------------------
%% Environments for code listings.
%%--------------------------------------------------

% We use the 'listings' package, with some small customizations.
% The most interesting customization: all TeX commands are available
% within comments. Comments are set in italics, keywords and strings
% don't get special treatment.

\lstset{language=C++,
        basicstyle=\small\CodeStyle,
        keywordstyle=,
        stringstyle=,
        xleftmargin=1em,
        showstringspaces=false,
        commentstyle=\itshape\rmfamily,
        columns=fullflexible,
        keepspaces=true,
        texcl=true}

% Our usual abbreviation for 'listings'.  Comments are in
% italics.  Arbitrary TeX commands can be used if they're
% surrounded by @ signs.
\newcommand{\CodeBlockSetup}{%
\lstset{escapechar=@, aboveskip=\parskip, belowskip=0pt,
        midpenalty=500, endpenalty=-50,
        emptylinepenalty=-250, semicolonpenalty=0,upquote=true}%
\renewcommand{\tcode}[1]{\textup{\CodeStylex{##1}}}
\renewcommand{\term}[1]{\textit{##1}}%
\renewcommand{\grammarterm}[1]{\gterm{##1}}%
}

\lstnewenvironment{codeblock}{\CodeBlockSetup}{}

% Left-align listings titles
\makeatletter
\def\lst@maketitle{\@makeleftcaption\lst@title@dropdelim}
\long\def\@makeleftcaption#1#2{%
  \vskip\abovecaptionskip
  \sbox\@tempboxa{#1: #2}%
  \ifdim \wd\@tempboxa >\hsize
    #1: #2\par
  \else
    \global \@minipagefalse
    \hb@xt@\hsize{%\hfil -- REMOVED
    \box\@tempboxa\hfil}%
  \fi
  \vskip\belowcaptionskip}%
\makeatother

\lstnewenvironment{codeblocktu}[1]{%
\lstset{title={%\parabullnum{Bullets1}{0pt}
#1:}}\CodeBlockSetup}{}

% An environment for command / program output that is not C++ code.
\lstnewenvironment{outputblock}{\lstset{language=}}{}

% A code block in which single-quotes are digit separators
% rather than character literals.
\lstnewenvironment{codeblockdigitsep}{
 \CodeBlockSetup
 \lstset{deletestring=[b]{'}}
}{}

% Permit use of '@' inside codeblock blocks (don't ask)
\makeatletter
\newcommand{\atsign}{@}
\makeatother

%%--------------------------------------------------
%% Indented text
%%--------------------------------------------------
\newenvironment{indented}[1][]
{\begin{indenthelper}[#1]\item\relax}
{\end{indenthelper}}

%%--------------------------------------------------
%% Library item descriptions
%%--------------------------------------------------
\lstnewenvironment{itemdecl}
{
 \lstset{escapechar=@,
 xleftmargin=0em,
 midpenalty=500,
 semicolonpenalty=-50,
 endpenalty=3000,
 aboveskip=2ex,
 belowskip=0ex	% leave this alone: it keeps these things out of the
				% footnote area
 }%
 \renewcommand{\tcode}[1]{\textup{\CodeStylex{##1}}}
}
{
}

\newenvironment{itemdescr}
{
 \begin{indented}[beginpenalty=3000, endpenalty=-300]}
{
 \end{indented}
}


%%--------------------------------------------------
%% Bnf environments
%%--------------------------------------------------
\newlength{\BnfIndent}
\setlength{\BnfIndent}{\leftmargini}
\newlength{\BnfInc}
\setlength{\BnfInc}{\BnfIndent}
\newlength{\BnfRest}
\setlength{\BnfRest}{2\BnfIndent}
\newcommand{\BnfNontermshape}{\small\color{grammar-gray}\sffamily\itshape}
\newcommand{\BnfReNontermshape}{\small\rmfamily\itshape}
\newcommand{\BnfTermshape}{\small\ttfamily\upshape}

\newenvironment{bnfbase}
 {
 \newcommand{\nontermdef}[1]{{\BnfNontermshape##1\itcorr}\indexgrammar{\idxgram{##1}}\textnormal{:}}
 \newcommand{\terminal}[1]{{\BnfTermshape ##1}}
 \renewcommand{\keyword}[1]{\terminal{##1}\indextext{\idxcode{##1}!keyword}}
 \renewcommand{\exposid}[1]{\terminal{\textit{##1}}}
 \renewcommand{\placeholder}[1]{\textrm{\textit{##1}}}
 \newcommand{\descr}[1]{\textnormal{##1}}
 \newcommand{\bnfindent}{\hspace*{\bnfindentfirst}}
 \newcommand{\bnfindentfirst}{\BnfIndent}
 \newcommand{\bnfindentinc}{\BnfInc}
 \newcommand{\bnfindentrest}{\BnfRest}
 \newcommand{\br}{\hfill\\*}
 \widowpenalties 1 10000
 \frenchspacing
 }
 {
 \nonfrenchspacing
 }

\newenvironment{simplebnf}
{
 \begin{bnfbase}
 \BnfNontermshape
 \begin{indented}[before*=\setlength{\rightmargin}{-\leftmargin}]
}
{
 \end{indented}
 \end{bnfbase}
}

\newenvironment{bnf}
{
 \begin{bnfbase}
 \begin{bnflist}
 \BnfNontermshape
 \item\relax
}
{
 \end{bnflist}
 \end{bnfbase}
}

\newenvironment{ncrebnf}
{
 \begin{bnfbase}
 \newcommand{\renontermdef}[1]{{\BnfReNontermshape##1\itcorr}\,\textnormal{::}}
 \begin{bnflist}
 \BnfReNontermshape
 \item\relax
}
{
 \end{bnflist}
 \end{bnfbase}
}

% non-copied versions of bnf environments
\let\ncsimplebnf\simplebnf
\let\endncsimplebnf\endsimplebnf
\let\ncbnf\bnf
\let\endncbnf\endbnf

%%--------------------------------------------------
%% Environment for imported graphics
%%--------------------------------------------------
% usage: \begin{importgraphic}{CAPTION}{TAG}{FILE}\end{importgraphic}
%        \importexample[VERTICAL OFFESET]{FILE}
%
% The filename is relative to the source/assets directory.

\newenvironment{importgraphic}[3]
{%
\newcommand{\cptn}{#1}
\newcommand{\lbl}{#2}
\begin{figure}[htp]\centering%
\includegraphics[scale=.35]{assets/#3}
}
{
\caption{\cptn \quad [fig:\lbl]}\label{fig:\lbl}%
\end{figure}}

\newcommand{\importexample}[2][-0.9pt]{\raisebox{#1}{\includegraphics{assets/#2}}}

%%--------------------------------------------------
%% Definitions section for "Terms and definitions"
%%--------------------------------------------------
\newcommand{\nocontentsline}[3]{}
\newcommand{\definition}[2]{%
\addxref{#2}%
\let\oldcontentsline\addcontentsline%
\let\addcontentsline\nocontentsline%
\ifcase\value{SectionDepth}
         \let\s=\section
      \or\let\s=\subsection
      \or\let\s=\subsubsection
      \or\let\s=\paragraph
      \or\let\s=\subparagraph
      \fi%
\s[#1]{\hfill[#2]}\vspace{-.3\onelineskip}\label{#2} \textbf{#1}\\*%
\let\addcontentsline\oldcontentsline%
}
\newcommand{\defncontext}[1]{\textlangle#1\textrangle}
\newnoteenvironment{defnote}{Note \arabic{defnote} to entry}{end note}

%!TEX root = std.tex
% Definitions of table environments

%%--------------------------------------------------
%% Table environments

% Set parameters for floating tables
\setcounter{totalnumber}{10}

% Base definitions for tables
\newenvironment{TableBase}
{
 \renewcommand{\tcode}[1]{\CodeStylex{##1}}
 \newcommand{\topline}{\hline}
 \newcommand{\capsep}{\hline\hline}
 \newcommand{\rowsep}{\hline}
 \newcommand{\bottomline}{\hline}

%% vertical alignment
 \newcommand{\rb}[1]{\raisebox{1.5ex}[0pt]{##1}}	% move argument up half a row

%% header helpers
 \newcommand{\hdstyle}[1]{\textbf{##1}}				% set header style
 \newcommand{\Head}[3]{\multicolumn{##1}{##2}{\hdstyle{##3}}}	% add title spanning multiple columns
 \newcommand{\lhdrx}[2]{\Head{##1}{|c}{##2}}		% set header for left column spanning #1 columns
 \newcommand{\chdrx}[2]{\Head{##1}{c}{##2}}			% set header for center column spanning #1 columns
 \newcommand{\rhdrx}[2]{\Head{##1}{c|}{##2}}		% set header for right column spanning #1 columns
 \newcommand{\ohdrx}[2]{\Head{##1}{|c|}{##2}}		% set header for only column spanning #1 columns
 \newcommand{\lhdr}[1]{\lhdrx{1}{##1}}				% set header for single left column
 \newcommand{\chdr}[1]{\chdrx{1}{##1}}				% set header for single center column
 \newcommand{\rhdr}[1]{\rhdrx{1}{##1}}				% set header for single right column
 \newcommand{\ohdr}[1]{\ohdrx{1}{##1}}
 \newcommand{\br}{\hfill\break}						% force newline within table entry

%% column styles
 \newcolumntype{x}[1]{>{\raggedright\let\\=\tabularnewline}p{##1}}      % word-wrapped ragged-right
                                                                        % column, width specified by #1

  % do not number bullets within tables
  \renewcommand{\labelitemi}{---}
  \renewcommand{\labelitemii}{---}
  \renewcommand{\labelitemiii}{---}
  \renewcommand{\labelitemiv}{---}
}
{
}

% floattablebase without TableBase, used for lib2dtab2base
\newenvironment{floattablebasex}[4]
{
 \begin{table}[#4]
 \caption{\label{tab:#2}#1 \quad [tab:#2]}
 \begin{center}
 \begin{tabular}{|#3|}
}
{
 \bottomline
 \end{tabular}
 \end{center}
 \end{table}
}


% General Usage: TITLE is the title of the table, XREF is the
% cross-reference for the table. LAYOUT is a sequence of column
% type specifiers (e.g., cp{1.0}c), without '|' for the left edge
% or right edge.

% usage: \begin{floattablebase}{TITLE}{XREF}{LAYOUT}{PLACEMENT}
% produces floating table, location determined within limits
% by LaTeX.
\newenvironment{floattablebase}[4]
{
 \begin{TableBase}
 \begin{floattablebasex}{#1}{#2}{#3}{#4}
}
{
 \end{floattablebasex}
 \end{TableBase}
}

% usage: \begin{floattable}{TITLE}{XREF}{LAYOUT}
% produces floating table, location determined within limits
% by LaTeX.
\newenvironment{floattable}[3]
{
 \begin{floattablebase}{#1}{#2}{#3}{htbp}
}
{
 \end{floattablebase}
}

% a column in a multicolfloattable (internal)
\newenvironment{mcftcol}{%
 \renewcommand{\columnbreak}{%
  \end{mcftcol} &
  \begin{mcftcol}
 }%
 \setlength{\tabcolsep}{0pt}%
 \begin{tabular}[t]{l}
}{
 \end{tabular}
}

% usage: \begin{multicolfloattable}{TITLE}{XREF}{LAYOUT}
% produces floating table, location determined within limits
% by LaTeX.
\newenvironment{multicolfloattable}[3]
{
 \begin{floattable}{#1}{#2}{#3}
 \topline
 \begin{mcftcol}
}
{
 \end{mcftcol} \\
 \end{floattable}
}

% usage: \begin{tokentable}{TITLE}{XREF}{HDR1}{HDR2}
% produces six-column table used for lists of replacement tokens;
% the columns are in pairs -- left-hand column has header HDR1,
% right hand column has header HDR2; pairs of columns are separated
% by vertical lines. Used in the "Alternative tokens" table.
\newenvironment{tokentable}[4]
{
 \begin{floattablebase}{#1}{#2}{cc|cc|cc}{htbp}
 \topline
 \hdstyle{#3}   &   \hdstyle{#4}    &
 \hdstyle{#3}   &   \hdstyle{#4}    &
 \hdstyle{#3}   &   \hdstyle{#4}    \\ \capsep
}
{
 \end{floattablebase}
}

% usage: \begin{libsumtabbase}{TITLE}{XREF}{HDR1}{HDR2}
% produces three-column table with column headers HDR1 and HDR2.
% Used in "Library Categories" table in standard, and used as
% base for other library summary tables.
\newenvironment{libsumtabbase}[5]
{
 \begin{floattable}{#2}{#3}{ll#1}
 \topline
 & \hdstyle{#4}	&	\hdstyle{#5}	\\ \capsep
}
{
 \end{floattable}
}

% usage: \begin{libsumtab}[LASTCOLUMN]{TITLE}{XREF}
% produces three-column table with column headers "Subclause" and "Header(s)".
% Used in "C++ Headers for Freestanding Implementations" table in standard.
\newenvironment{libsumtab}[3][l]
{
 \begin{libsumtabbase}{#1}{#2}{#3}{Subclause}{Header}
}
{
 \end{libsumtabbase}
}

% usage: \begin{concepttable}{TITLE}{XREF}{LAYOUT}
% produces table at current location
\newenvironment{concepttable}[3]
{
 \begin{floattablebase}{#1}{#2}{#3}{!htb}
}
{
 \end{floattablebase}
}

% usage: \begin{oldconcepttable}{NAME}{EXTRA}{XREF}{LAYOUT}
% produces table at current location
\newenvironment{oldconcepttable}[4]
{
 \indextext{\idxoldconcept{#1}}%
 \begin{concepttable}{\oldconcept{#1} requirements#2}{#3}{#4}
}
{
 \end{concepttable}
}

% usage: \begin{simpletypetable}{TITLE}{XREF}{LAYOUT}
% produces table at current location
\newenvironment{simpletypetable}[3]
{
 \begin{floattablebase}{#1}{#2}{#3}{!htb}
}
{
 \end{floattablebase}
}

% usage: \begin{LongTable}{TITLE}{XREF}{LAYOUT}
% produces table that handles page breaks sensibly.
% WARNING: Putting two of these on the same page
% does not break sensibly. Avoid this for short tables.
\newenvironment{LongTable}[3]
{
 \newcommand{\continuedcaption}{\caption[]{#1 (continued)}}
 \begin{TableBase}
 \begin{longtable}{|#3|}
 \caption{#1 \quad [tab:#2]}\label{tab:#2}
}
{
 \bottomline
 \end{longtable}
 \end{TableBase}
}

% usage: \begin{libreqtabN}{TITLE}{XREF}
% produces an N-column breakable table. Used in
% most of the library Clauses for requirements tables.
% Example at "Position type requirements" in the standard.

\newenvironment{libreqtab2}[2]
{
 \begin{LongTable}
 {#1}{#2}
 {lx{.55\hsize}}
}
{
 \end{LongTable}
}

\newenvironment{shortlibreqtab2}[2]
{
 \begin{floattable}
 {#1}{#2}
 {lx{.55\hsize}}
}
{
 \end{floattable}
}

\newenvironment{libreqtab2a}[2]
{
 \begin{LongTable}
 {#1}{#2}
 {x{.30\hsize}x{.64\hsize}}
}
{
 \end{LongTable}
}

\newenvironment{libreqtab3}[2]
{
 \begin{LongTable}
 {#1}{#2}
 {x{.28\hsize}x{.18\hsize}x{.43\hsize}}
}
{
 \end{LongTable}
}

\newenvironment{libreqtab3a}[2]
{
 \begin{LongTable}
 {#1}{#2}
 {x{.28\hsize}x{.33\hsize}x{.29\hsize}}
}
{
 \end{LongTable}
}

\newenvironment{libreqtab3b}[2]
{
 \begin{LongTable}
 {#1}{#2}
 {x{.40\hsize}x{.25\hsize}x{.25\hsize}}
}
{
 \end{LongTable}
}

\newenvironment{libreqtab3e}[2]
{
 \begin{LongTable}
 {#1}{#2}
 {x{.38\hsize}x{.27\hsize}x{.25\hsize}}
}
{
 \end{LongTable}
}

\newenvironment{libreqtab3f}[2]
{
 \begin{LongTable}
 {#1}{#2}
 {x{.35\hsize}x{.28\hsize}x{.29\hsize}}
}
{
 \end{LongTable}
}

\newenvironment{libreqtab4a}[2]
{
 \begin{LongTable}
 {#1}{#2}
 {x{.14\hsize}x{.30\hsize}x{.30\hsize}x{.14\hsize}}
}
{
 \end{LongTable}
}

\newenvironment{libreqtab4b}[3][LongTable]
{
 \def\libreqtabenv{#1}
 \begin{\libreqtabenv}
 {#2}{#3}
 {x{.13\hsize}x{.15\hsize}x{.29\hsize}x{.27\hsize}}
}
{
 \end{\libreqtabenv}
}

\newenvironment{libreqtab4c}[2]
{
 \begin{LongTable}
 {#1}{#2}
 {x{.16\hsize}x{.21\hsize}x{.21\hsize}x{.30\hsize}}
}
{
 \end{LongTable}
}

\newenvironment{libreqtab4d}[2]
{
 \begin{LongTable}
 {#1}{#2}
 {x{.22\hsize}x{.22\hsize}x{.30\hsize}x{.15\hsize}}
}
{
 \end{LongTable}
}

\newenvironment{libreqtab5}[2]
{
 \begin{LongTable}
 {#1}{#2}
 {x{.14\hsize}x{.14\hsize}x{.20\hsize}x{.20\hsize}x{.14\hsize}}
}
{
 \end{LongTable}
}

% usage: \begin{libtab2}{TITLE}{XREF}{LAYOUT}{HDR1}{HDR2}
% produces two-column table with column headers HDR1 and HDR2.
% Used in "seekoff positioning" in the standard.
\newenvironment{libtab2}[5]
{
 \begin{floattable}
 {#1}{#2}{#3}
 \topline
 \lhdr{#4}	&	\rhdr{#5}	\\ \capsep
}
{
 \end{floattable}
}

% usage: \begin{longlibtab2}{TITLE}{XREF}{LAYOUT}{HDR1}{HDR2}
% produces two-column table with column headers HDR1 and HDR2.
\newenvironment{longlibtab2}[5]
{
 \begin{LongTable}{#1}{#2}{#3}
 \\ \topline
 \lhdr{#4}	&	\rhdr{#5}	\\ \capsep
 \endfirsthead
 \continuedcaption\\
 \topline
 \lhdr{#4}	&	\rhdr{#5}	\\ \capsep
 \endhead
}
{
  \end{LongTable}
}

% usage: \begin{LibEffTab}{TITLE}{XREF}{HDR2}{WD2}
% produces a two-column table with left column header "Element"
% and right column header HDR2, right column word-wrapped with
% width specified by WD2.
\newenvironment{LibEffTab}[4]
{
 \begin{libtab2}{#1}{#2}{lp{#4}}{Element}{#3}
}
{
 \end{libtab2}
}

% Same as LibEffTab except that it uses a long table.
\newenvironment{longLibEffTab}[4]
{
 \begin{longlibtab2}{#1}{#2}{lp{#4}}{Element}{#3}
}
{
 \end{longlibtab2}
}

% usage: \begin{libefftab}{TITLE}{XREF}
% produces a two-column effects table with right column
% header "Effect(s) if set", width 4.5 in. Used in "fmtflags effects"
% table in standard.
\newenvironment{libefftab}[2]
{
 \begin{LibEffTab}{#1}{#2}{Effect(s) if set}{4.5in}
}
{
 \end{LibEffTab}
}

% Same as libefftab except that it uses a long table.
\newenvironment{longlibefftab}[2]
{
 \begin{longLibEffTab}{#1}{#2}{Effect(s) if set}{4.5in}
}
{
 \end{longLibEffTab}
}

% usage: \begin{libefftabmean}{TITLE}{XREF}
% produces a two-column effects table with right column
% header "Meaning", width 4.5 in. Used in "seekdir effects"
% table in standard.
\newenvironment{libefftabmean}[2]
{
 \begin{LibEffTab}{#1}{#2}{Meaning}{4.5in}
}
{
 \end{LibEffTab}
}

% usage: \begin{libefftabvalue}{TITLE}{XREF}
% produces a two-column effects table with right column
% header "Value", width 3 in. Used in "basic_ios::init() effects"
% table in standard.
\newenvironment{libefftabvalue}[2]
{
 \begin{LibEffTab}{#1}{#2}{Value}{3in}
}
{
 \end{LibEffTab}
}

% Same as libefftabvalue except that it uses a long table and a
% slightly wider column.
\newenvironment{longlibefftabvalue}[2]
{
 \begin{longLibEffTab}{#1}{#2}{Value}{3.5in}
}
{
 \end{longLibEffTab}
}

% usage: \begin{liberrtab}{TITLE}{XREF} produces a two-column table
% with left column header ``Value'' and right header "Error
% condition", width 4.5 in. Used in regex Clause in the TR.

\newenvironment{liberrtab}[2]
{
 \begin{libtab2}{#1}{#2}{lp{4.5in}}{Value}{Error condition}
}
{
 \end{libtab2}
}

% Like liberrtab except that it uses a long table.
\newenvironment{longliberrtab}[2]
{
 \begin{longlibtab2}{#1}{#2}{lp{4.5in}}{Value}{Error condition}
}
{
 \end{longlibtab2}
}

% usage: \begin{lib2dtab2base}{TITLE}{XREF}{HDR1}{HDR2}{WID1}{WID2}{WID3}
% produces a table with one heading column followed by 2 data columns.
% used for 2D requirements tables, such as optional::operator= effects
% tables.
\newenvironment{lib2dtab2base}[7]
{
 %% no lines in the top-left cell, and leave a gap around the headers
 %% FIXME: I tried to use hhline here, but it doesn't appear to support
 %% the join between the leftmost top header and the topmost left header,
 %% so we fake it with an empty row and column.
 \newcommand{\topline}{\cline{3-4}}
 \newcommand{\rowsep}{\cline{1-1}\cline{3-4}}
 \newcommand{\capsep}{
  \topline
  \multicolumn{4}{c}{}\\[-0.8\normalbaselineskip]
  \rowsep
 }
 \newcommand{\bottomline}{\rowsep}
 \newcommand{\hdstyle}[1]{\textbf{##1}}
 \newcommand{\rowhdr}[1]{\hdstyle{##1}&}
 \newcommand{\colhdr}[1]{\multicolumn{1}{|>{\centering}m{#6}|}{\hdstyle{##1}}}
 \begin{floattablebasex}
 {#1}{#2}
 {>{\centering}m{#5}|@{}p{0.2\normalbaselineskip}@{}|m{#6}|m{#7} }
 {htbp}
 %% table header
 \topline
 \multicolumn{1}{c}{}&&\colhdr{#3}&\colhdr{#4}\\
 \capsep
}
{
 \end{floattablebasex}
}

\newenvironment{lib2dtab2}[4]{
 \begin{lib2dtab2base}{#1}{#2}{#3}{#4}{1.2in}{1.8in}{1.8in}
}{
 \end{lib2dtab2base}
}

%% Cross reference
\renewcommand{\xref}{\textsc{See also:}\xspace}
\renewcommand{\tref}[1]{}
\renewcommand{\iref}[1]{}
\renewcommand{\indexhdr}[1]{}

\usepackage{amssymb,graphicx,stackengine,xcolor}
\protected\def\ucr{\scalebox{1}{\stackinset{c}{}{c}{-.2pt}{%
      \textcolor{white}{\sffamily\bfseries\small ?}}{%
      \rotatebox{45}{$\blacksquare$}}}}

\newenvironment{wording}{
  \chapterstyle{cppstd}
  \newcounter{scratchChapter}\setcounter{scratchChapter}{\value{chapter}}
  \settocdepth{part}
  \renewcommand\thesection{\ucr.\arabic{section}}
  \renewcommand\thesubsection{\makebox{$\ucr$}.\makebox{$\ucr$}.\arabic{subsection}}
} {
  \setcounter{chapter}{\value{scratchChapter}}
  \settocdepth{section}

}

\newenvironment{addedcode}
{
\color{addclr}
\begin{codeblock}
}
{
\end{codeblock}
\color{black}
}

\newenvironment{removedcode}
{
\color{remclr}
\begin{codeblock}
}
{
\end{codeblock}
\color{black}
}

\newcommand{\todo}[1]{}
\renewcommand{\todo}[1]{{\color{red} TODO: {#1}}}


% %% --------------------------------------------------
% %% fix interaction between hyperref and other
% %% commands
% \pdfstringdefDisableCommands{\def\smaller#1{#1}}
% \pdfstringdefDisableCommands{\def\textbf#1{#1}}
% \pdfstringdefDisableCommands{\def\raisebox#1{}}
% \pdfstringdefDisableCommands{\def\hspace#1{}}



%%--------------------------------------------------
%% add special hyphenation rules
\hyphenation{tem-plate ex-am-ple in-put-it-er-a-tor name-space name-spaces non-zero}

%%--------------------------------------------------
%% turn off all ligatures inside \texttt
\DisableLigatures{encoding = T1, family = tt*}

%% --------------------------------------------------
%% configuration
%!TEX root = std.tex
%%--------------------------------------------------
%% Version numbers
% \newcommand{\docno}{Dxxxx}
% \newcommand{\prevdocno}{N4917}
% \newcommand{\cppver}{202002L}

%% Release date
% \newcommand{\reldate}{\today}

%% Library chapters
% \newcommand{\firstlibchapter}{support}
% \newcommand{\lastlibchapter}{thread}


\settocdepth{chapter}
\usepackage{minted}
\usepackage{fontspec}
\setromanfont{Source Serif Pro}
\setsansfont{Source Sans Pro}
% \setmonofont{Source Code Pro}

\begin{document}
\title{std::optional<T\&>}
\author{
  Steve Downey \small<\href{mailto:sdowney@gmail.com}{sdowney@gmail.com}> \\
  Peter Sommerlad \small<\href{mailto:peter.cpp@sommerlad.ch}{peter.cpp@sommerlad.ch}> \\
}
\date{} %unused. Type date explicitly below.
\maketitle

\begin{flushright}
  \begin{tabular}{ll}
    Document \#: & P2988R8 \\
    Date: & \today \\
    Project: & Programming Language C++ \\
    Audience: & LEWG
  \end{tabular}
\end{flushright}

\begin{abstract}
  We propose to fix a hole intentionally left in \tcode{std::optional} ---

  An optional over a reference such that the post condition on assignment is independent of the engaged state, always producing a rebound reference, and assigning a \tcode{U} to a \tcode{T} is disallowed by \tcode{static_assert} if a \tcode{U} can not be bound to a \tcode{T\&}.
\end{abstract}

\tableofcontents*

\chapter*{Changes Since Last Version}

\begin{itemize}
\item \textbf{Changes since R7}
  \begin{itemize}
  \item Wording mandates/constraint fixes
  \item Hash on T\& pulled out
  \item Notes on wording rendering
  \item ``Fix'' make_optional<T\&>
  \end{itemize}
\item \textbf{Changes since R6}
  \begin{itemize}
  \item strike refref specialization
  \item add converting assignment operator
  \item add converting in place constructor
  \end{itemize}
\item \textbf{Changes since R5}
  \begin{itemize}
  \item refref specialization
  \item fix monadic constraints on base template
  \end{itemize}
\item \textbf{Changes since R4}
  \begin{itemize}
  \item feature test macro
  \item value_or updates from P3091
  \end{itemize}
\item \textbf{Changes since R3}
  \begin{itemize}
  \item make_optional discussion - always value
  \item value_or discussion - always value
  \end{itemize}
\end{itemize}

\chapter{Comparison table}
\section{Using a raw pointer result for an element search function}

This is the convention the C++ core guidelines suggest, to use a raw pointer for representing optional non-owning references.
However, there is a user-required check against \tcode{nullptr}, no type safety meaning no safety against mis-interpreting such a raw pointer, for example by using pointer arithmetic on it.

\begin{tabular}{ lr }
  \begin{minipage}[t]{0.45\columnwidth}
    \begin{minted}[fontsize=\small]{c++}
      Cat* cat = find_cat("Fido");
      if (cat!=nullptr) { return doit(*cat); }


    \end{minted}
  \end{minipage}
  &
    \begin{minipage}[t]{0.45\columnwidth}
      \begin{minted}[fontsize=\small]{c++}
        std::optional<Cat&> cat = find_cat("Fido");
        return cat.and_then(doit);

      \end{minted}
    \end{minipage}
\end{tabular}

\section{returning result of an element search function via a (smart) pointer}

The disadvantage here is that \tcode{std::experimental::observer_ptr<T>} is both non-standard and not well named, therefore this example uses \tcode{shared_ptr} that would have the advantage of avoiding dangling through potential lifetime extension.
However, on the downside is still the explicit checks against the \tcode{nullptr} on the client side, failing so risks undefined behavior.

  \begin{tabular}{ lr }
  \begin{minipage}[t]{0.45\columnwidth}
    \begin{minted}[fontsize =\small]{c++}
std::shared_ptr<Cat> cat = find_cat("Fido");
if (cat != nullptr) {/* ... */}

    \end{minted}
  \end{minipage}
  &
    \begin{minipage}[t]{0.45\columnwidth}
      \begin{minted}[fontsize=\small]{c++}
std::optional<Cat&> cat = find_cat("Fido");
cat.and_then([](Cat& thecat){/* ... */}

      \end{minted}
    \end{minipage}
  \end{tabular}
  \section{returning result of an element search function via an iterator}

  This might be the obvious choice, for example, for associative containers, especially since their iterator stability guarantees.
  However, returning such an iterator will leak the underlying container type as well necessarily requires one to know the sentinel of the container to check for the not-found case.

  \begin{tabular}{ lr }
  \begin{minipage}[t]{0.45\columnwidth}
    \begin{minted}[fontsize=\small]{c++}
std::map<std::string, Cat>::iterator cat
        = find_cat("Fido");
if (cat != theunderlyingmap.end()){/* ... */}

    \end{minted}
  \end{minipage}
  &
    \begin{minipage}[t]{0.45\columnwidth}
      \begin{minted}[fontsize=\small]{c++}
std::optional<Cat&> cat
        = find_cat("Fido");
cat.and_then([](Cat& thecat){/* ... */}

      \end{minted}
    \end{minipage}
  \end{tabular}

  \section{Using an optional<T*> as a substitute for optional<T\&>}

This approach adds another level of indirection and requires two checks to take a definite action.

  \begin{tabular}{ lr }
  \begin{minipage}[t]{0.45\columnwidth}
    \begin{minted}[fontsize=\small]{c++}
//Mutable optional
std::optional<Cat*> c = find_cat("Fido");
if (c) {
    if (*c) {
      *c.value() = Cat("Fynn", color::orange);
    }
}

    \end{minted}
  \end{minipage}
  &
    \begin{minipage}[t]{0.45\columnwidth}
      \begin{minted}[fontsize=\small]{c++}
std::optional<Cat&> c = find_cat("Fido");
if (c) {
  *c = Cat("Fynn", color::orange);
}

//or

o.transform([](Cat& c){
  c = Cat("Fynn", color::orange);
});
        \end{minted}
      \end{minipage}
\end{tabular}

\chapter{Motivation}
Other than the standard library's implementation of optional, optionals holding references are common. The desire for such a feature is well understood, and many optional types in commonly used libraries provide it, with the semantics proposed here.
One standard library implementation already provides an implementation of \tcode{std::optional<T\&>} but disables its use, because the standard forbids it.

The research in JeanHeyd Meneide's _References for Standard Library Vocabulary Types - an optional case study._ \cite{P1683R0} shows conclusively that rebind semantics are the only safe semantic as assign through on engaged is too bug-prone. Implementations that attempt assign-through are abandoned. The standard library should follow existing practice and supply an \tcode{optional<T\&>} that rebinds on assignment.

Additional background reading on \tcode{optional<T\&>} can be found in JeanHeyd Meneide's article _To Bind and Loose a Reference_ \cite{REFBIND}.

In freestanding environments or for safety-critical libraries, an optional type over references is important to implement containers, that otherwise as the standard library either would cause undefined behavior when accessing an non-available element, throw an exception, or silently create the element. Returning a plain pointer for such an optional reference, as the core guidelines suggest, is a non-type-safe solution and doesn't protect in any way from accessing an non-existing element by a \tcode{nullptr} de-reference. In addition, the monadic APIs of \tcode{std::optional} makes is especially attractive by streamlining client code receiving such an optional reference, in contrast to a pointer that requires an explicit nullptr check and de-reference.

There is a principled reason not to provide a partial specialization over \tcode{T\&} as the semantics are in some ways subtly different than the primary template. Assignment may have side-effects not present in the primary, which has pure value semantics. However, I argue this is misleading, as reference semantics often has side-effects. The proposed semantic is similar to what an \tcode{optional<std::reference_wrapper<T>>} provides, with much greater usability.

There are well motivated suggestions that perhaps instead of an \tcode{optional<T\&>} there should be an \tcode{optional_ref<T>} that is an independent primary template. This proposal rejects that, because we need a policy over all sum types as to how reference semantics should work, as optional is a variant over T and monostate. That the library sum type can not express the same range of types as the product type, tuple, is an increasing problem as we add more types logically equivalent to a variant. The template types \tcode{optional} and \tcode{expected} should behave as extensions of \tcode{variant<T, monostate>} and \tcode{variant<T, E>}, or we lose the ability to reason about generic types.

That we can't guarantee from \tcode{std::tuple<Args...>} (product type) that \tcode{std::variant<Args...>} (sum type) is valid, is a problem, and one that reflection can't solve. A language sum type could, but we need agreement on the semantics.

The semantics of a variant with a reference are as if it holds the address of the referent when referring to that referent. All other semantics are worse. Not being able to express a variant<T\&> is inconsistent, hostile, and strictly worse than disallowing it.

Thus, we expect future papers to propose \tcode{std::expected<T\&,E>} and \tcode{std::variant} with the ability to hold references.
The latter can be used as an iteration type over \tcode{std::tuple} elements.


\chapter{Design}

The design is straightforward. The \tcode{optional<T\&>} holds a pointer to the underlying object of type \tcode{T}, or \tcode{nullptr} if the optional is disengaged. The implementation is simple, especially with C++20 and up techniques, using concept constraints. As the held pointer is a primitive regular type with reference semantics, many operations can be defaulted and are \tcode{noexcept} by nature. See \cite{Downey_smd_optional_optional_T} and \cite{rawgithu58:online}. The \tcode{optional<T\&>} implementation is less than 200 lines of code, much of it the monadic functions with identical textual implementations with different signatures and different overloads being called.

In place construction is not supported as it would just be a way of providing immediate life-time issues.

\section{Relational Operations}

The definitions of the relational operators are the same as for the base template. Interoperable comparisons between T and optional<T\&> work as expected. This is not true for the boost optional<T\&>.

\section{make_optional}
\begin{removedblock}
\sout{Because of existing code, \tcode{make_optional<T\&>} must return optional<T> rather than optional<T\&>. Returning optional<T\&> is consistent and defensible, and a few optional implementations in production make this choice. It is, however, quite easy to construct a make_optional expression that deduces a different category causing possibly dangerous changes to code.}

\sout{There was some discussion about using library technology to allow selection of the reference overload via the literal spelling \tcode{make_optional<int\&>}. There was anti-consensus to do so. There are existing instances of that spelling that today return an \tcode{optional<T>}, although it is very likely these are mistakes or possibly other optionals. The spelling \tcode{optional<T\&>\{\}} is acceptable as there is no multi-argument emplacement version as there is no location to construct such an instance.}
\end{removedblock}

\begin{addedblock}
With further research, the existing uses of make_optional<X\&> seem to be primarily test cases, and deliberate use seems to be exceedingly rare in the wild. Reflector review was much more positive about removing the misleading ability to create an \tcode{optional<X>} via \tcode{make_optional<X\&>(x)}. In addition, the multiple argument forms can be used to attempt to construct a optional that contains a reference, but this becomes ill formed because of existing mandates at the type level. In order to preserve existing behavior, where make_optional is not well formed if it constructs a reference, changes to \tcode{make_optional} should be made.

Adding a non-type template parameter as the first template parameter to the single argument \tcode{make_optional} and mandating that the multi-argument version not request a reference type as the parameter, will diagnose mistaken use of \tcode{make_optional} and preserve the existing behavior.

Since construction of an object in order to make a reference to it to construct an optional containing a reference would always dangle, there do not seem to be any use cases for the multi-argument or initializer list forms of make_optional for reference types, and the constructor form seems to satisfy all cases for single argument construction of a optional containing a reference, there does not seem to be a need for a factory function for optional over reference.

\end{addedblock}

There was also discussion of using \tcode{std::reference_wrapper} to indicate reference use, in analogy with std::tuple. Unfortunately there are existing uses of optional over reference_wrapper as a workaround for lack of reference specialization, and it would be a breaking change for such code.

\section{Trivial construction}
Construction of \tcode{optional<T\&>} should be trivial, because it is straightforward to implement, and \tcode{optional<T>} is trivial. Boost is not.

\section{Value Category Affects value()}
For several implementations there are distinct overloads for functions depending on value category, with the same implementation. However, this makes it very easy to accidentally steal from the underlying referred to object. Value category should be shallow. Thanks to many people for pointing this out. If ``Deducing \tcode{this}'' had been used, the problem would have been much more subtle in code review.

\section{Shallow vs Deep const}

There is some implementation divergence in optionals about deep const for \tcode{optional<T\&>}. That is, can the referred to \tcode{int} be modified through a \tcode{const optional<int\&>}. Does \tcode{operator->()} return an \tcode{int*} or a \tcode{const int*}, and does \tcode{operator*()} return an \tcode{int\&} or a \tcode{const int\&}. I believe it is overall more defensible if the \tcode{const} is shallow as it would be for a \tcode{struct ref {int * p;}} where the constness of the struct ref does not affect if the p pointer can be written through. This is consistent with the rebinding behavior being proposed.

Where deeper constness is desired, \tcode{optional<const T\&>} would prevent non const access to the underlying object.

\section{Conditional Explicit}
As in the base template, \tcode{explicit} is made conditional on the type used to construct the optional. \tcode{explicit(!std::is_convertible_v<U, T>)}. This is not present in boost::optional, leading to differences in construction between braced initialization and = that can be surprising.

\section{value_or}
\removed{Have \tcode{value_or} return a \tcode{T\&}. Check that the supplied value can be bound to a T\&.}

After extensive discussion, it seems there is no particularly wonderful solution for \tcode{value_or} that does not involve a time machine. Implementations of optionals that support reference semantics diverge over the return type, and the current one is arguably wrong, and should use something based on \tcode{common_reference_type}, which of course did not exist when \tcode{optional} was standardized.

The weak consensus is to return a \tcode{T} from \tcode{optional<T\&>::value_or} as this is least likely to cause issues. There was at least one strong objection to this choice, but all other choices had more objections. The author intends to propose free functions \tcode{reference_or}, \tcode{value_or}, \tcode{or_invoke}, and \tcode{yield_if} over all types modeling optional-like, \tcode{concept std::maybe}, in the next revision of \cite{P1255R12}. This would cover \tcode{optional}, \tcode{expected}, and pointer types.

Having \tcode{value_or} return by value also allows the common case of using a literal as the alternative to be expressed concisely.

\section{in_place_t construction}
The reference specialization allows a limited form of in_place construction where the argument can be converted to the appropriate reference without creation of a temporary. As the reference specialization is non-owning, there is no ``place'' for a temporary to be constructed that will not dangle. For cases where the lifetime of the constructed object would match the lifetime of the optional, the temporary can be constructed explicitly, instead.

\section{Converting assignment}
A similarly limited converting assignment operator is provided for cases where an optional<U> has a value or refers to a value which can be converted to a T\& without construction of a temporary. In particular, converting an optional<T\&> to an optional<T const\&> is supported.

\section{Compiler Explorer Playground}

See \url{https://compiler-explorer.com/z/zKqE3sn87} for an updated playground with relevant Google Test functions and various optional implementations made available for cross reference including a flattened in-place version of the reference implementation.

\chapter{Principles for Reification of Design}

Optional must never construct a temporary, or knowingly take the address of an temporary or part of an temporary.

It is always presumed safe to copy the pointer value from an optional, since by induction, it is not dangling.

Optional has no storage, so should never construct a T, it may convert a U to a T, so long as that conversion does not create a temporary.

Constructors that would convert from temporary are marked deleted. They should be sufficiently constrained that it was the correct choice and there is no more general, less constrained constructor that would not have created a dangling pointer.

Failure to compile either by ambiguity or no eligible constructors in the overload set is preferable to optional being responsible for use after free or dangling.

Assignment is always from an optional, which may have been an implicit construction. The assignment cannot throw, the construction/conversion may. The assignment may therefore need annotation converting the rhs if that constructor was explicit. This must not be necessary in the default case of creating an optional reference to an lvalue of the same type.

The model for the constraints and mandates for \tcode{optional<T\&>} is taken from \tcode{std::tuple} over reference types. The type \tcode{std::tuple} takes the most care of types in the standard library in dealing with creation of temporaries.

As \tcode{optional} is designed to be converting, to create instances from arguments that can be used to create the underlying type, constructors should be explicit only where the operations used to create the pointer or the notional reference would be or are explicit.



\section{Construction from temporary}

We disallow construction of \tcode{optional<T\&>} from any type U in which:
\begin{itemize}
\item the constructor body will create a temporary and bind it to a reference.
\item a const lvalue reference would be bound to rvalue.
\end{itemize}

An example of the first case would be construction \tcode{optional<std::string const\&>} from \tcode{char const*}. These cases always dangle.

An example of the second case would be a construction \tcode{std::optional<std::string const\&>} from temporary \tcode{std::string}.

Prohibiting the second case does prevent some safe uses of the optional as the function parameter.

Given:

\tcode{void process(std::optional<std::string const\&> arg);}

This will make a \tcode{process(std::string("sdfd"))} invocation ill-formed, despite the arg being safe to use from within the function body.

This deviates from the design of the ``view'' parameters type, like \tcode{std::string_view} or \tcode{std::span}. However, we believe that this is the right choice due to the following:

\begin{itemize}
\item
  Only a subset of cases would be working. As an illustration the very similar \tcode{process("text")} invocation is ill-formed, due to always being dangling.

\item
  Such design leads to the detection of reference to temporaries or local variables when \tcode{optional<T const\&>} is used as the return type.

  \begin{tabular}{ lr }
  \begin{minipage}[t]{0.45\columnwidth}
    \begin{minted}[fontsize=\small]{c++}
std::optional<std::string const&> getValue() {
  std::string localString;
  return localString; // Ill-formed.
  std::optional<std::string> localOptionalString;
  return localOptionalString; // ill-formed
}
        \end{minted}
      \end{minipage}
\end{tabular}

One of the main motivational examples of \tcode{optional<T const\&>} is return from a lookup function, and eliminating dangling in such cases outweighs parameter cases.

We are very grateful to Arthur O'Dwyer for his work on \cite{P2266R3} P2266R3 Simpler implicit move accepted in C++23, which makes it possible to implement this correctly.

\item
  We provide behavior consistent with \tcode{reference_wrapper<T const>}, that disallows binding to xvalues. We believe that \tcode{reference_wrapper<T>} is closer in spirit to \tcode{optional<T\&>} than any view type. It certainly shares some of the features.
\end{itemize}

\section{Deleting dangling overloads}

To achieve the dangling safety expressed before, the constructor is marked deleted if it would lead to binding of the reference to temporary or the xvalue.
However, deleted constructors are still considered to be candidates during overload resolution, leading to ambiguity in the following examples:

\begin{tabular}{ lr }
  \begin{minipage}[t]{0.45\columnwidth}
    \begin{minted}[fontsize=\small]{c++}
void process(std::optional<std::string const&>);
void process(std::optional<char const* const&>);

void test() {
  char const* cstr = “Text”;
  std::string s = cstr;
  process(s); // Picks, optional<std::string const&> overload
  process(cstr); // Ambiguous, but only std::optional<char const* const&> is not dangling
}
\end{minted}
\end{minipage}
\end{tabular}


During the reflector discussion, an option of an alternate design was presented, where the dangling overload would be constrained, and eliminated from the overload set.

We strongly oppose changing this behavior, as:
\begin{itemize}
\item We think that it is impossible to detect temporary binding to xvalue in such a design.

\item The behavior we propose is consistent with the behavior for optional for object types

\begin{tabular}{ lr }
  \begin{minipage}[t]{0.45\columnwidth}
    \begin{minted}[fontsize=\small]{c++}
void processVal(std::optional<std::string>);
void processVal(std::optional<char const*>);

void test() {
  char const* cstr = "Text";
  std::string s = cstr;
  processVal(s); // Picks std::string overload
  processVal(cstr); // Ambiguous
}
\end{minted}
\end{minipage}
\end{tabular}
\end{itemize}
As language in general treats functions accepting by value and by const reference in the same manner during overload resolution, we believe achieving this consistency is a feature.

The design that was introduced by \tcode{std::tuple}, and \tcode{std::pair}, for references, is followed, where the detection of dangling does not affect the results of overload resolution and instead makes a call that would dangle be ill-formed and diagnosed.

\section{Assignment of optional<T\&>}

In the case of \tcode{optional<T\&>}, any assignment operation is equivalent to assigning a pointer, and there is no observable difference between:
using converting assignment from \tcode{U\&\&} or \tcode{optional<U>}
constructing temporary \tcode{optional<T\&>}, and then assigning it to it.

This observation allows us to provide only copy-assignment for \tcode{optional<T\&>}, instead of a set of converting assignments, that would need to replicate the signatures of constructors and their constraints. Assignment from any other value is handled by first implicitly constructing \tcode{optional<T\&>} and then using copy-assignment. Move-assignment is the same as copy-assignment, since only pointer copy is involved.


\chapter{Proposal}

Add an lvalue reference specialization for the std::optional template.

\chapter{Wording}

The wording here cross references and adopts the wording in \cite{P3091R2}.


\begin{wording}

\rSec1[optional]{Optional objects}

\rSec2[optional.general]{General}

\pnum
Subclause~\ref{optional} describes class template \tcode{optional} that represents
optional objects.
An \defn{optional object} is an
object that contains the storage for another object and manages the lifetime of
this contained object, if any. The contained object may be initialized after
the optional object has been initialized, and may be destroyed before the
optional object has been destroyed. The initialization state of the contained
object is tracked by the optional object.

\rSec2[optional.syn]{Header \tcode{<optional>} synopsis}

\indexheader{optional}%
\begin{codeblock}
// mostly freestanding
#include <compare>              // see \ref{compare.syn}

namespace std {
  // \ref{optional.optional}, class template \tcode{optional}
  template<class T>
    class optional;                                                     // partially freestanding

  template<class T>
    constexpr bool ranges::enable_view<optional<T>> = true;
  template<class T>
    constexpr auto format_kind<optional<T>> = range_format::disabled;

  template<class T>
    concept @\defexposconcept{is-derived-from-optional}@ = requires(const T& t) {       // \expos
      []<class U>(const optional<U>&){ }(t);
    };

  // \ref{optional.nullopt}, no-value state indicator
  struct nullopt_t{@\seebelow@};
  inline constexpr nullopt_t nullopt(@\unspec@);

  // \ref{optional.bad.access}, class \tcode{bad_optional_access}
  class bad_optional_access;

  // \ref{optional.relops}, relational operators
  template<class T, class U>
    constexpr bool operator==(const optional<T>&, const optional<U>&);
  template<class T, class U>
    constexpr bool operator!=(const optional<T>&, const optional<U>&);
  template<class T, class U>
    constexpr bool operator<(const optional<T>&, const optional<U>&);
  template<class T, class U>
    constexpr bool operator>(const optional<T>&, const optional<U>&);
  template<class T, class U>
    constexpr bool operator<=(const optional<T>&, const optional<U>&);
  template<class T, class U>
    constexpr bool operator>=(const optional<T>&, const optional<U>&);
  template<class T, @\libconcept{three_way_comparable_with}@<T> U>
    constexpr compare_three_way_result_t<T, U>
      operator<=>(const optional<T>&, const optional<U>&);

  // \ref{optional.nullops}, comparison with \tcode{nullopt}
  template<class T> constexpr bool operator==(const optional<T>&, nullopt_t) noexcept;
  template<class T>
    constexpr strong_ordering operator<=>(const optional<T>&, nullopt_t) noexcept;

  // \ref{optional.comp.with.t}, comparison with \tcode{T}
  template<class T, class U> constexpr bool operator==(const optional<T>&, const U&);
  template<class T, class U> constexpr bool operator==(const T&, const optional<U>&);
  template<class T, class U> constexpr bool operator!=(const optional<T>&, const U&);
  template<class T, class U> constexpr bool operator!=(const T&, const optional<U>&);
  template<class T, class U> constexpr bool operator<(const optional<T>&, const U&);
  template<class T, class U> constexpr bool operator<(const T&, const optional<U>&);
  template<class T, class U> constexpr bool operator>(const optional<T>&, const U&);
  template<class T, class U> constexpr bool operator>(const T&, const optional<U>&);
  template<class T, class U> constexpr bool operator<=(const optional<T>&, const U&);
  template<class T, class U> constexpr bool operator<=(const T&, const optional<U>&);
  template<class T, class U> constexpr bool operator>=(const optional<T>&, const U&);
  template<class T, class U> constexpr bool operator>=(const T&, const optional<U>&);
  template<class T, class U>
      requires (!@\exposconcept{is-derived-from-optional}@<U>) && @\libconcept{three_way_comparable_with}@<T, U>
    constexpr compare_three_way_result_t<T, U>
      operator<=>(const optional<T>&, const U&);

  // \ref{optional.specalg}, specialized algorithms
  template<class T>
    constexpr void swap(optional<T>&, optional<T>&) noexcept(@\seebelow@);

  template<class T>
    constexpr optional<@\seebelow@> make_optional(T&&);
  template<class T, class... Args>
    constexpr optional<T> make_optional(Args&&... args);
  template<class T, class U, class... Args>
    constexpr optional<T> make_optional(initializer_list<U> il, Args&&... args);

  // \ref{optional.hash}, hash support
  template<class T> struct hash;
  template<class T> struct hash<optional<T>>;
}
\end{codeblock}

\rSec2[optional.optional]{Class template \tcode{optional}}

\rSec3[optional.optional.general]{General}

\indexlibraryglobal{optional}%
\indexlibrarymember{value_type}{optional}%
\begin{codeblock}
namespace std {
  template<class T>
  class optional {
  public:
    using value_type     = T;
    using iterator       = @\impdefnc@;              // see~\ref{optional.iterators}
    using const_iterator = @\impdefnc@;              // see~\ref{optional.iterators}

    // \ref{optional.ctor}, constructors
    constexpr optional() noexcept;
    constexpr optional(nullopt_t) noexcept;
    constexpr optional(const optional&);
    constexpr optional(optional&&) noexcept(@\seebelow@);
    template<class... Args>
      constexpr explicit optional(in_place_t, Args&&...);
    template<class U, class... Args>
      constexpr explicit optional(in_place_t, initializer_list<U>, Args&&...);
    template<class U = T>
      constexpr explicit(@\seebelow@) optional(U&&);
    template<class U>
      constexpr explicit(@\seebelow@) optional(const optional<U>&);
    template<class U>
      constexpr explicit(@\seebelow@) optional(optional<U>&&);

    // \ref{optional.dtor}, destructor
    constexpr ~optional();

    // \ref{optional.assign}, assignment
    constexpr optional& operator=(nullopt_t) noexcept;
    constexpr optional& operator=(const optional&);
    constexpr optional& operator=(optional&&) noexcept(@\seebelow@);
    template<class U = T> constexpr optional& operator=(U&&);
    template<class U> constexpr optional& operator=(const optional<U>&);
    template<class U> constexpr optional& operator=(optional<U>&&);
    template<class... Args> constexpr T& emplace(Args&&...);
    template<class U, class... Args> constexpr T& emplace(initializer_list<U>, Args&&...);

    // \ref{optional.swap}, swap
    constexpr void swap(optional&) noexcept(@\seebelow@);

    // \ref{optional.iterators}, iterator support
    constexpr iterator begin() noexcept;
    constexpr const_iterator begin() const noexcept;
    constexpr iterator end() noexcept;
    constexpr const_iterator end() const noexcept;

    // \ref{optional.observe}, observers
    constexpr const T* operator->() const noexcept;
    constexpr T* operator->() noexcept;
    constexpr const T& operator*() const & noexcept;
    constexpr T& operator*() & noexcept;
    constexpr T&& operator*() && noexcept;
    constexpr const T&& operator*() const && noexcept;
    constexpr explicit operator bool() const noexcept;
    constexpr bool has_value() const noexcept;
    constexpr const T& value() const &;                                 // freestanding-deleted
    constexpr T& value() &;                                             // freestanding-deleted
    constexpr T&& value() &&;                                           // freestanding-deleted
    constexpr const T&& value() const &&;                               // freestanding-deleted
    template<class U> constexpr T value_or(U&&) const &;
    template<class U> constexpr T value_or(U&&) &&;

    // \ref{optional.monadic}, monadic operations
    template<class F> constexpr auto and_then(F&& f) &;
    template<class F> constexpr auto and_then(F&& f) &&;
    template<class F> constexpr auto and_then(F&& f) const &;
    template<class F> constexpr auto and_then(F&& f) const &&;
    template<class F> constexpr auto transform(F&& f) &;
    template<class F> constexpr auto transform(F&& f) &&;
    template<class F> constexpr auto transform(F&& f) const &;
    template<class F> constexpr auto transform(F&& f) const &&;
    template<class F> constexpr optional or_else(F&& f) &&;
    template<class F> constexpr optional or_else(F&& f) const &;

    // \ref{optional.mod}, modifiers
    constexpr void reset() noexcept;

  private:
    T *val;         // \expos
  };

  template<class T>
    optional(T) -> optional<T>;
}
\end{codeblock}

\pnum
Any instance of \tcode{optional<T>} at any given time either contains a value or does not contain a value.
When an instance of \tcode{optional<T>} \defnx{contains a value}{contains a value!\idxcode{optional}},
it means that an object of type \tcode{T}, referred to as the optional object's \defnx{contained value}{contained value!\idxcode{optional}},
is allocated within the storage of the optional object.
Implementations are not permitted to use additional storage, such as dynamic memory, to allocate its contained value.
When an object of type \tcode{optional<T>} is contextually converted to \tcode{bool},
the conversion returns \tcode{true} if the object contains a value;
otherwise the conversion returns \tcode{false}.

\pnum
When an \tcode{optional<T>} object contains a value,
member \tcode{val} points to the contained value.

\pnum
\tcode{T} shall be a type
other than \cv{} \tcode{in_place_t} or \cv{} \tcode{nullopt_t}
that meets the \oldconcept{Destructible} requirements (\tref{cpp17.destructible}).

\rSec3[optional.ctor]{Constructors}

\pnum
The exposition-only variable template \exposid{converts-from-any-cvref}
is used by some constructors for \tcode{optional}.
\begin{codeblock}
template<class T, class W>
constexpr bool @\exposid{converts-from-any-cvref}@ =  // \expos
  disjunction_v<is_constructible<T, W&>, is_convertible<W&, T>,
                is_constructible<T, W>, is_convertible<W, T>,
                is_constructible<T, const W&>, is_convertible<const W&, T>,
                is_constructible<T, const W>, is_convertible<const W, T>>;
\end{codeblock}

\indexlibraryctor{optional}%
\begin{itemdecl}
constexpr optional() noexcept;
constexpr optional(nullopt_t) noexcept;
\end{itemdecl}

\begin{itemdescr}
\pnum
\ensures
\tcode{*this} does not contain a value.

\pnum
\remarks
No contained value is initialized.
For every object type \tcode{T} these constructors are constexpr constructors\iref{dcl.constexpr}.
\end{itemdescr}

\indexlibraryctor{optional}%
\begin{itemdecl}
constexpr optional(const optional& rhs);
\end{itemdecl}

\begin{itemdescr}
\pnum
\effects
If \tcode{rhs} contains a value, direct-non-list-initializes the contained value
with \tcode{*rhs}.

\pnum
\ensures
\tcode{rhs.has_value() == this->has_value()}.

\pnum
\throws
Any exception thrown by the selected constructor of \tcode{T}.

\pnum
\remarks
This constructor is defined as deleted unless
\tcode{is_copy_constructible_v<T>} is \tcode{true}.
If \tcode{is_trivially_copy_constructible_v<T>} is \tcode{true},
this constructor is trivial.
\end{itemdescr}

\indexlibraryctor{optional}%
\begin{itemdecl}
constexpr optional(optional&& rhs) noexcept(@\seebelow@);
\end{itemdecl}

\begin{itemdescr}
\pnum
\constraints
\tcode{is_move_constructible_v<T>} is \tcode{true}.

\pnum
\effects
If \tcode{rhs} contains a value, direct-non-list-initializes the contained value
with \tcode{std::move(*rhs)}.
\tcode{rhs.has_value()} is unchanged.

\pnum
\ensures
\tcode{rhs.has_value() == this->has_value()}.

\pnum
\throws
Any exception thrown by the selected constructor of \tcode{T}.

\pnum
\remarks
The exception specification is equivalent to
\tcode{is_nothrow_move_constructible_v<T>}.
If \tcode{is_trivially_move_constructible_v<T>} is \tcode{true},
this constructor is trivial.
\end{itemdescr}

\indexlibraryctor{optional}%
\begin{itemdecl}
template<class... Args> constexpr explicit optional(in_place_t, Args&&... args);
\end{itemdecl}

\begin{itemdescr}
\pnum
\constraints
\tcode{is_constructible_v<T, Args...>} is \tcode{true}.

\pnum
\effects
Direct-non-list-initializes the contained value with \tcode{std::forward<Args>(args)...}.

\pnum
\ensures
\tcode{*this} contains a value.

\pnum
\throws
Any exception thrown by the selected constructor of \tcode{T}.

\pnum
\remarks
If \tcode{T}'s constructor selected for the initialization is a constexpr constructor, this constructor is a constexpr constructor.
\end{itemdescr}

\indexlibraryctor{optional}%
\begin{itemdecl}
template<class U, class... Args>
  constexpr explicit optional(in_place_t, initializer_list<U> il, Args&&... args);
\end{itemdecl}

\begin{itemdescr}
\pnum
\constraints
\tcode{is_constructible_v<T, initializer_list<U>\&, Args...>} is \tcode{true}.

\pnum
\effects
Direct-non-list-initializes the contained value with \tcode{il, std::forward<Args>(args)...}.

\pnum
\ensures
\tcode{*this} contains a value.

\pnum
\throws
Any exception thrown by the selected constructor of \tcode{T}.

\pnum
\remarks
If \tcode{T}'s constructor selected for the initialization is a constexpr constructor, this constructor is a constexpr constructor.
\end{itemdescr}

\indexlibraryctor{optional}%
\begin{itemdecl}
template<class U = T> constexpr explicit(@\seebelow@) optional(U&& v);
\end{itemdecl}

\begin{itemdescr}
\pnum
\constraints
\begin{itemize}
\item \tcode{is_constructible_v<T, U>} is \tcode{true},
\item \tcode{is_same_v<remove_cvref_t<U>, in_place_t>} is \tcode{false},
\item \tcode{is_same_v<remove_cvref_t<U>, optional>} is \tcode{false}, and
\item if \tcode{T} is \cv{} \tcode{bool},
\tcode{remove_cvref_t<U>} is not a specialization of \tcode{optional}.
\end{itemize}

\pnum
\effects
Direct-non-list-initializes the contained value with \tcode{std::forward<U>(v)}.

\pnum
\ensures
\tcode{*this} contains a value.

\pnum
\throws
Any exception thrown by the selected constructor of \tcode{T}.

\pnum
\remarks
If \tcode{T}'s selected constructor is a constexpr constructor,
this constructor is a constexpr constructor.
The expression inside \keyword{explicit} is equivalent to:
\begin{codeblock}
!is_convertible_v<U, T>
\end{codeblock}
\end{itemdescr}

\indexlibraryctor{optional}%
\begin{itemdecl}
template<class U> constexpr explicit(@\seebelow@) optional(const optional<U>& rhs);
\end{itemdecl}

\begin{itemdescr}
\pnum
\constraints
\begin{itemize}
\item \tcode{is_constructible_v<T, const U\&>} is \tcode{true}, and
\item if \tcode{T} is not \cv{} \tcode{bool},
\tcode{\exposid{converts-from-any-cvref}<T, optional<U>>} is \tcode{false}.
\end{itemize}

\pnum
\effects
If \tcode{rhs} contains a value,
direct-non-list-initializes the contained value with \tcode{*rhs}.

\pnum
\ensures
\tcode{rhs.has_value() == this->has_value()}.

\pnum
\throws
Any exception thrown by the selected constructor of \tcode{T}.

\pnum
\remarks
The expression inside \keyword{explicit} is equivalent to:
\begin{codeblock}
!is_convertible_v<const U&, T>
\end{codeblock}
\end{itemdescr}

\indexlibraryctor{optional}%
\begin{itemdecl}
template<class U> constexpr explicit(@\seebelow@) optional(optional<U>&& rhs);
\end{itemdecl}

\begin{itemdescr}
\pnum
\constraints
\begin{itemize}
\item \tcode{is_constructible_v<T, U>} is \tcode{true}, and
\item if \tcode{T} is not \cv{} \tcode{bool},
\tcode{\exposid{converts-from-any-cvref}<T, optional<U>>} is \tcode{false}.
\end{itemize}

\pnum
\effects
If \tcode{rhs} contains a value,
direct-non-list-initializes the contained value with \tcode{std::move(*rhs)}.
\tcode{rhs.has_value()} is unchanged.

\pnum
\ensures
\tcode{rhs.has_value() == this->has_value()}.

\pnum
\throws
Any exception thrown by the selected constructor of \tcode{T}.

\pnum
\remarks
The expression inside \keyword{explicit} is equivalent to:
\begin{codeblock}
!is_convertible_v<U, T>
\end{codeblock}
\end{itemdescr}

\rSec3[optional.dtor]{Destructor}

\indexlibrarydtor{optional}%
\begin{itemdecl}
constexpr ~optional();
\end{itemdecl}

\begin{itemdescr}
\pnum
\effects
If \tcode{is_trivially_destructible_v<T> != true} and \tcode{*this} contains a value, calls
\begin{codeblock}
val->T::~T()
\end{codeblock}

\pnum
\remarks
If \tcode{is_trivially_destructible_v<T>} is \tcode{true}, then this destructor is trivial.
\end{itemdescr}

\rSec3[optional.assign]{Assignment}

\indexlibrarymember{operator=}{optional}%
\begin{itemdecl}
constexpr optional<T>& operator=(nullopt_t) noexcept;
\end{itemdecl}

\begin{itemdescr}
\pnum
\effects
If \tcode{*this} contains a value, calls \tcode{val->T::\~T()} to destroy the contained value; otherwise no effect.

\pnum
\ensures
\tcode{*this} does not contain a value.

\pnum
\returns
\tcode{*this}.
\end{itemdescr}

\indexlibrarymember{operator=}{optional}%
\begin{itemdecl}
constexpr optional<T>& operator=(const optional& rhs);
\end{itemdecl}

\begin{itemdescr}
\pnum
\effects
See \tref{optional.assign.copy}.
\begin{lib2dtab2}{\tcode{optional::operator=(const optional\&)} effects}{optional.assign.copy}
{\tcode{*this} contains a value}
{\tcode{*this} does not contain a value}

\rowhdr{\tcode{rhs} contains a value} &
assigns \tcode{*rhs} to the contained value &
direct-non-list-initializes the contained value with \tcode{*rhs} \\
\rowsep

\rowhdr{\tcode{rhs} does not contain a value} &
destroys the contained value by calling \tcode{val->T::\~T()} &
no effect \\
\end{lib2dtab2}

\pnum
\ensures
\tcode{rhs.has_value() == this->has_value()}.

\pnum
\returns
\tcode{*this}.

\pnum
\remarks
If any exception is thrown, the result of the expression \tcode{this->has_value()} remains unchanged.
If an exception is thrown during the call to \tcode{T}'s copy constructor, no effect.
If an exception is thrown during the call to \tcode{T}'s copy assignment,
the state of its contained value is as defined by the exception safety guarantee of \tcode{T}'s copy assignment.
This operator is defined as deleted unless
\tcode{is_copy_constructible_v<T>} is \tcode{true} and
\tcode{is_copy_assignable_v<T>} is \tcode{true}.
If \tcode{is_trivially_copy_constructible_v<T> \&\&}
\tcode{is_trivially_copy_assignable_v<T> \&\&}
\tcode{is_trivially_destructible_v<T>} is \tcode{true},
this assignment operator is trivial.
\end{itemdescr}

\indexlibrarymember{operator=}{optional}%
\begin{itemdecl}
constexpr optional& operator=(optional&& rhs) noexcept(@\seebelow@);
\end{itemdecl}

\begin{itemdescr}
\pnum
\constraints
\tcode{is_move_constructible_v<T>} is \tcode{true} and
\tcode{is_move_assignable_v<T>} is \tcode{true}.

\pnum
\effects
See \tref{optional.assign.move}.
The result of the expression \tcode{rhs.has_value()} remains unchanged.
\begin{lib2dtab2}{\tcode{optional::operator=(optional\&\&)} effects}{optional.assign.move}
{\tcode{*this} contains a value}
{\tcode{*this} does not contain a value}

\rowhdr{\tcode{rhs} contains a value} &
assigns \tcode{std::move(*rhs)} to the contained value &
direct-non-list-initializes the contained value with \tcode{std::move(*rhs)} \\
\rowsep

\rowhdr{\tcode{rhs} does not contain a value} &
destroys the contained value by calling \tcode{val->T::\~T()} &
no effect \\
\end{lib2dtab2}

\pnum
\ensures
\tcode{rhs.has_value() == this->has_value()}.

\pnum
\returns
\tcode{*this}.

\pnum
\remarks
The exception specification is equivalent to:
\begin{codeblock}
is_nothrow_move_assignable_v<T> && is_nothrow_move_constructible_v<T>
\end{codeblock}

\pnum
If any exception is thrown, the result of the expression \tcode{this->has_value()} remains unchanged.
If an exception is thrown during the call to \tcode{T}'s move constructor,
the state of \tcode{*rhs.val} is determined by the exception safety guarantee of \tcode{T}'s move constructor.
If an exception is thrown during the call to \tcode{T}'s move assignment,
the state of \tcode{*val} and \tcode{*rhs.val} is determined by the exception safety guarantee of \tcode{T}'s move assignment.
If \tcode{is_trivially_move_constructible_v<T> \&\&}
\tcode{is_trivially_move_assignable_v<T> \&\&}
\tcode{is_trivially_destructible_v<T>} is \tcode{true},
this assignment operator is trivial.
\end{itemdescr}

\indexlibrarymember{operator=}{optional}%
\begin{itemdecl}
template<class U = T> constexpr optional<T>& operator=(U&& v);
\end{itemdecl}

\begin{itemdescr}
\pnum
\constraints
\tcode{is_same_v<remove_cvref_t<U>, optional>} is \tcode{false},
\tcode{conjunction_v<is_scalar<T>, is_same<T, decay_t<U>>>} is \tcode{false},
\tcode{is_constructible_v<T, U>} is \tcode{true}, and
\tcode{is_assignable_v<T\&, U>} is \tcode{true}.

\pnum
\effects
If \tcode{*this} contains a value, assigns \tcode{std::forward<U>(v)} to the contained value; otherwise direct-non-list-initializes the contained value with \tcode{std::forward<U>(v)}.

\pnum
\ensures
\tcode{*this} contains a value.

\pnum
\returns
\tcode{*this}.

\pnum
\remarks
If any exception is thrown, the result of the expression \tcode{this->has_value()} remains unchanged. If an exception is thrown during the call to \tcode{T}'s constructor, the state of \tcode{v} is determined by the exception safety guarantee of \tcode{T}'s constructor. If an exception is thrown during the call to \tcode{T}'s assignment, the state of \tcode{*val} and \tcode{v} is determined by the exception safety guarantee of \tcode{T}'s assignment.
\end{itemdescr}

\indexlibrarymember{operator=}{optional}%
\begin{itemdecl}
template<class U> constexpr optional<T>& operator=(const optional<U>& rhs);
\end{itemdecl}

\begin{itemdescr}
\pnum
\constraints
\begin{itemize}
\item \tcode{is_constructible_v<T, const U\&>} is \tcode{true},
\item \tcode{is_assignable_v<T\&, const U\&>} is \tcode{true},
\item \tcode{\exposid{converts-from-any-cvref}<T, optional<U>>} is \tcode{false},
\item \tcode{is_assignable_v<T\&, optional<U>\&>} is \tcode{false},
\item \tcode{is_assignable_v<T\&, optional<U>\&\&>} is \tcode{false},
\item \tcode{is_assignable_v<T\&, const optional<U>\&>} is \tcode{false}, and
\item \tcode{is_assignable_v<T\&, const optional<U>\&\&>} is \tcode{false}.
\end{itemize}

\pnum
\effects
See \tref{optional.assign.copy.templ}.
\begin{lib2dtab2}{\tcode{optional::operator=(const optional<U>\&)} effects}{optional.assign.copy.templ}
{\tcode{*this} contains a value}
{\tcode{*this} does not contain a value}

\rowhdr{\tcode{rhs} contains a value} &
assigns \tcode{*rhs} to the contained value &
direct-non-list-initializes the contained value with \tcode{*rhs} \\
\rowsep

\rowhdr{\tcode{rhs} does not contain a value} &
destroys the contained value by calling \tcode{val->T::\~T()} &
no effect \\
\end{lib2dtab2}

\pnum
\ensures
\tcode{rhs.has_value() == this->has_value()}.

\pnum
\returns
\tcode{*this}.

\pnum
\remarks
If any exception is thrown,
the result of the expression \tcode{this->has_value()} remains unchanged.
If an exception is thrown during the call to \tcode{T}'s constructor,
the state of \tcode{*rhs.val} is determined by
the exception safety guarantee of \tcode{T}'s constructor.
If an exception is thrown during the call to \tcode{T}'s assignment,
the state of \tcode{*val} and \tcode{*rhs.val} is determined by
the exception safety guarantee of \tcode{T}'s assignment.
\end{itemdescr}

\indexlibrarymember{operator=}{optional}%
\begin{itemdecl}
template<class U> constexpr optional<T>& operator=(optional<U>&& rhs);
\end{itemdecl}

\begin{itemdescr}
\pnum
\constraints
\begin{itemize}
\item \tcode{is_constructible_v<T, U>} is \tcode{true},
\item \tcode{is_assignable_v<T\&, U>} is \tcode{true},
\item \tcode{\exposid{converts-from-any-cvref}<T, optional<U>>} is \tcode{false},
\item \tcode{is_assignable_v<T\&, optional<U>\&>} is \tcode{false},
\item \tcode{is_assignable_v<T\&, optional<U>\&\&>} is \tcode{false},
\item \tcode{is_assignable_v<T\&, const optional<U>\&>} is \tcode{false}, and
\item \tcode{is_assignable_v<T\&, const optional<U>\&\&>} is \tcode{false}.
\end{itemize}

\pnum
\effects
See \tref{optional.assign.move.templ}.
The result of the expression \tcode{rhs.has_value()} remains unchanged.
\begin{lib2dtab2}{\tcode{optional::operator=(optional<U>\&\&)} effects}{optional.assign.move.templ}
{\tcode{*this} contains a value}
{\tcode{*this} does not contain a value}

\rowhdr{\tcode{rhs} contains a value} &
assigns \tcode{std::move(*rhs)} to the contained value &
direct-non-list-initializes the contained value with \tcode{std::move(*rhs)} \\
\rowsep

\rowhdr{\tcode{rhs} does not contain a value} &
destroys the contained value by calling \tcode{val->T::\~T()} &
no effect \\
\end{lib2dtab2}

\pnum
\ensures
\tcode{rhs.has_value() == this->has_value()}.

\pnum
\returns
\tcode{*this}.

\pnum
\remarks
If any exception is thrown,
the result of the expression \tcode{this->has_value()} remains unchanged.
If an exception is thrown during the call to \tcode{T}'s constructor,
the state of \tcode{*rhs.val} is determined by
the exception safety guarantee of \tcode{T}'s constructor.
If an exception is thrown during the call to \tcode{T}'s assignment,
the state of \tcode{*val} and \tcode{*rhs.val} is determined by
the exception safety guarantee of \tcode{T}'s assignment.
\end{itemdescr}

\indexlibrarymember{emplace}{optional}%
\begin{itemdecl}
template<class... Args> constexpr T& emplace(Args&&... args);
\end{itemdecl}

\begin{itemdescr}
\pnum
\mandates
\tcode{is_constructible_v<T, Args...>} is \tcode{true}.

\pnum
\effects
Calls \tcode{*this = nullopt}. Then direct-non-list-initializes the contained value
with \tcode{std::forward\brk{}<Args>(args)...}.

\pnum
\ensures
\tcode{*this} contains a value.

\pnum
\returns
A reference to the new contained value.

\pnum
\throws
Any exception thrown by the selected constructor of \tcode{T}.

\pnum
\remarks
If an exception is thrown during the call to \tcode{T}'s constructor, \tcode{*this} does not contain a value, and the previous \tcode{*val} (if any) has been destroyed.
\end{itemdescr}

\indexlibrarymember{emplace}{optional}%
\begin{itemdecl}
template<class U, class... Args> constexpr T& emplace(initializer_list<U> il, Args&&... args);
\end{itemdecl}

\begin{itemdescr}
\pnum
\constraints
\tcode{is_constructible_v<T, initializer_list<U>\&, Args...>} is \tcode{true}.

\pnum
\effects
Calls \tcode{*this = nullopt}. Then direct-non-list-initializes the contained value with
\tcode{il, std::\brk{}forward<Args>(args)...}.

\pnum
\ensures
\tcode{*this} contains a value.

\pnum
\returns
A reference to the new contained value.

\pnum
\throws
Any exception thrown by the selected constructor of \tcode{T}.

\pnum
\remarks
If an exception is thrown during the call to \tcode{T}'s constructor, \tcode{*this} does not contain a value, and the previous \tcode{*val} (if any) has been destroyed.
\end{itemdescr}

\rSec3[optional.swap]{Swap}

\indexlibrarymember{swap}{optional}%
\begin{itemdecl}
constexpr void swap(optional& rhs) noexcept(@\seebelow@);
\end{itemdecl}

\begin{itemdescr}
\pnum
\mandates
\tcode{is_move_constructible_v<T>} is \tcode{true}.

\pnum
\expects
\tcode{T} meets the \oldconcept{Swappable} requirements\iref{swappable.requirements}.

\pnum
\effects
See \tref{optional.swap}.
\begin{lib2dtab2}{\tcode{optional::swap(optional\&)} effects}{optional.swap}
{\tcode{*this} contains a value}
{\tcode{*this} does not contain a value}

\rowhdr{\tcode{rhs} contains a value} &
calls \tcode{swap(*(*this), *rhs)} &
direct-non-list-initializes the contained value of \tcode{*this}
with \tcode{std::move(*rhs)},
followed by \tcode{rhs.val->T::\~T()};
postcondition is that \tcode{*this} contains a value and \tcode{rhs} does not contain a value \\
\rowsep

\rowhdr{\tcode{rhs} does not contain a value} &
direct-non-list-initializes the contained value of \tcode{rhs}
with \tcode{std::move(*(*this))},
followed by \tcode{val->T::\~T()};
postcondition is that \tcode{*this} does not contain a value and \tcode{rhs} contains a value &
no effect \\
\end{lib2dtab2}

\pnum
\throws
Any exceptions thrown by the operations in the relevant part of \tref{optional.swap}.

\pnum
\remarks
The exception specification is equivalent to:
\begin{codeblock}
is_nothrow_move_constructible_v<T> && is_nothrow_swappable_v<T>
\end{codeblock}

\pnum
If any exception is thrown, the results of the expressions \tcode{this->has_value()} and \tcode{rhs.has_value()} remain unchanged.
If an exception is thrown during the call to function \tcode{swap},
the state of \tcode{*val} and \tcode{*rhs.val} is determined by the exception safety guarantee of \tcode{swap} for lvalues of \tcode{T}.
If an exception is thrown during the call to \tcode{T}'s move constructor,
the state of \tcode{*val} and \tcode{*rhs.val} is determined by the exception safety guarantee of \tcode{T}'s move constructor.
\end{itemdescr}

\rSec3[optional.iterators]{Iterator support}

\indexlibrarymember{iterator}{optional}%
\indexlibrarymember{const_iterator}{optional}%
\begin{itemdecl}
using iterator       = @\impdef@;
using const_iterator = @\impdef@;
\end{itemdecl}

\begin{itemdescr}
\pnum
These types
model \libconcept{contiguous_iterator}\iref{iterator.concept.contiguous},
meet the \oldconcept{RandomAccessIterator} requirements\iref{random.access.iterators}, and
meet the requirements for constexpr iterators\iref{iterator.requirements.general},
with value type \tcode{remove_cv_t<T>}.
The reference type is \tcode{T\&} for \tcode{iterator} and
\tcode{const T\&} for \tcode{const_iterator}.

\pnum
All requirements on container iterators\iref{container.reqmts} apply to
\tcode{optional::iterator} and \tcode{optional::\linebreak{}const_iterator} as well.

\pnum
Any operation that initializes or destroys the contained value of an optional object invalidates all iterators into that object.
\end{itemdescr}

\indexlibrarymember{begin}{optional}%
\begin{itemdecl}
constexpr iterator begin() noexcept;
constexpr const_iterator begin() const noexcept;
\end{itemdecl}

\begin{itemdescr}
\pnum
\returns
If \tcode{has_value()} is \tcode{true},
an iterator referring to the contained value.
Otherwise, a past-the-end iterator value.
\end{itemdescr}

\indexlibrarymember{end}{optional}%
\begin{itemdecl}
constexpr iterator end() noexcept;
constexpr const_iterator end() const noexcept;
\end{itemdecl}

\begin{itemdescr}
\pnum
\returns
\tcode{begin() + has_value()}.
\end{itemdescr}

\rSec3[optional.observe]{Observers}

\indexlibrarymember{operator->}{optional}%
\begin{itemdecl}
constexpr const T* operator->() const noexcept;
constexpr T* operator->() noexcept;
\end{itemdecl}

\begin{itemdescr}
\pnum
\expects
\tcode{*this} contains a value.

\pnum
\returns
\tcode{val}.

\pnum
\remarks
These functions are constexpr functions.
\end{itemdescr}

\indexlibrarymember{operator*}{optional}%
\begin{itemdecl}
constexpr const T& operator*() const & noexcept;
constexpr T& operator*() & noexcept;
\end{itemdecl}

\begin{itemdescr}
\pnum
\expects
\tcode{*this} contains a value.

\pnum
\returns
\tcode{*val}.

\pnum
\remarks
These functions are constexpr functions.
\end{itemdescr}

\indexlibrarymember{operator*}{optional}%
\begin{itemdecl}
constexpr T&& operator*() && noexcept;
constexpr const T&& operator*() const && noexcept;
\end{itemdecl}

\begin{itemdescr}
\pnum
\expects
\tcode{*this} contains a value.

\pnum
\effects
Equivalent to: \tcode{return std::move(*val);}
\end{itemdescr}

\indexlibrarymember{operator bool}{optional}%
\begin{itemdecl}
constexpr explicit operator bool() const noexcept;
\end{itemdecl}

\begin{itemdescr}
\pnum
\returns
\tcode{true} if and only if \tcode{*this} contains a value.

\pnum
\remarks
This function is a constexpr function.
\end{itemdescr}

\indexlibrarymember{has_value}{optional}%
\begin{itemdecl}
constexpr bool has_value() const noexcept;
\end{itemdecl}

\begin{itemdescr}
\pnum
\returns
\tcode{true} if and only if \tcode{*this} contains a value.

\pnum
\remarks
This function is a constexpr function.
\end{itemdescr}

\indexlibrarymember{value}{optional}%
\begin{itemdecl}
constexpr const T& value() const &;
constexpr T& value() &;
\end{itemdecl}

\begin{itemdescr}
\pnum
\effects
Equivalent to:
\begin{codeblock}
return has_value() ? *val : throw bad_optional_access();
\end{codeblock}
\end{itemdescr}

\indexlibrarymember{value}{optional}%
\begin{itemdecl}
constexpr T&& value() &&;
constexpr const T&& value() const &&;
\end{itemdecl}

\begin{itemdescr}

\pnum
\effects
Equivalent to:
\begin{codeblock}
return has_value() ? std::move(*val) : throw bad_optional_access();
\end{codeblock}
\end{itemdescr}

\indexlibrarymember{value_or}{optional}%
\begin{itemdecl}
template<class U> constexpr T value_or(U&& v) const &;
\end{itemdecl}

\begin{itemdescr}
\pnum
\mandates
\tcode{is_copy_constructible_v<T> \&\& is_convertible_v<U\&\&, T>} is \tcode{true}.

\pnum
\effects
Equivalent to:
\begin{codeblock}
return has_value() ? **this : static_cast<T>(std::forward<U>(v));
\end{codeblock}
\end{itemdescr}

\indexlibrarymember{value_or}{optional}%
\begin{itemdecl}
template<class U> constexpr T value_or(U&& v) &&;
\end{itemdecl}

\begin{itemdescr}
\pnum
\mandates
\tcode{is_move_constructible_v<T> \&\& is_convertible_v<U\&\&, T>} is \tcode{true}.

\pnum
\effects
Equivalent to:
\begin{codeblock}
return has_value() ? std::move(**this) : static_cast<T>(std::forward<U>(v));
\end{codeblock}
\end{itemdescr}

\rSec3[optional.monadic]{Monadic operations}

\indexlibrarymember{and_then}{optional}
\begin{itemdecl}
template<class F> constexpr auto and_then(F&& f) &;
template<class F> constexpr auto and_then(F&& f) const &;
\end{itemdecl}

\begin{itemdescr}
\pnum
Let \tcode{U} be \tcode{invoke_result_t<F, decltype(*val)>}.

\pnum
\mandates
\tcode{remove_cvref_t<U>} is a specialization of \tcode{optional}.

\pnum
\effects
Equivalent to:
\begin{codeblock}
if (*this) {
  return invoke(std::forward<F>(f), *val);
} else {
  return remove_cvref_t<U>();
}
\end{codeblock}
\end{itemdescr}

\indexlibrarymember{and_then}{optional}
\begin{itemdecl}
template<class F> constexpr auto and_then(F&& f) &&;
template<class F> constexpr auto and_then(F&& f) const &&;
\end{itemdecl}

\begin{itemdescr}
\pnum
Let \tcode{U} be \tcode{invoke_result_t<F, decltype(std::move(*val))>}.

\pnum
\mandates
\tcode{remove_cvref_t<U>} is a specialization of \tcode{optional}.

\pnum
\effects
Equivalent to:
\begin{codeblock}
if (*this) {
  return invoke(std::forward<F>(f), std::move(*val));
} else {
  return remove_cvref_t<U>();
}
\end{codeblock}
\end{itemdescr}

\indexlibrarymember{transform}{optional}
\begin{itemdecl}
template<class F> constexpr auto transform(F&& f) &;
template<class F> constexpr auto transform(F&& f) const &;
\end{itemdecl}

\begin{itemdescr}
\pnum
Let \tcode{U} be \tcode{remove_cv_t<invoke_result_t<F, decltype(*val)>>}.

\pnum
\mandates
\tcode{U} is a non-array object type
other than \tcode{in_place_t} or \tcode{nullopt_t}.
The declaration
\begin{codeblock}
U u(invoke(std::forward<F>(f), *val));
\end{codeblock}
is well-formed for some invented variable \tcode{u}.
\begin{note}
There is no requirement that \tcode{U} is movable\iref{dcl.init.general}.
\end{note}

\pnum
\returns
If \tcode{*this} contains a value, an \tcode{optional<U>} object
whose contained value is direct-non-list-initialized with
\tcode{invoke(std::forward<F>(f), *val)};
otherwise, \tcode{optional<U>()}.
\end{itemdescr}

\indexlibrarymember{transform}{optional}
\begin{itemdecl}
template<class F> constexpr auto transform(F&& f) &&;
template<class F> constexpr auto transform(F&& f) const &&;
\end{itemdecl}

\begin{itemdescr}
\pnum
Let \tcode{U} be
\tcode{remove_cv_t<invoke_result_t<F, decltype(std::move(*val))>>}.

\pnum
\mandates
\tcode{U} is a non-array object type
other than \tcode{in_place_t} or \tcode{nullopt_t}.
The declaration
\begin{codeblock}
U u(invoke(std::forward<F>(f), std::move(*val)));
\end{codeblock}
is well-formed for some invented variable \tcode{u}.
\begin{note}
There is no requirement that \tcode{U} is movable\iref{dcl.init.general}.
\end{note}

\pnum
\returns
If \tcode{*this} contains a value, an \tcode{optional<U>} object
whose contained value is direct-non-list-initialized with
\tcode{invoke(std::forward<F>(f), std::move(*val))};
otherwise, \tcode{optional<U>()}.
\end{itemdescr}

\indexlibrarymember{or_else}{optional}
\begin{itemdecl}
template<class F> constexpr optional or_else(F&& f) const &;
\end{itemdecl}

\begin{itemdescr}
\pnum
\constraints
\tcode{F} models \tcode{\libconcept{invocable}<>} and
\tcode{T} models \libconcept{copy_constructible}.

\pnum
\mandates
\tcode{is_same_v<remove_cvref_t<invoke_result_t<F>>, optional>} is \tcode{true}.

\pnum
\effects
Equivalent to:
\begin{codeblock}
if (*this) {
  return *this;
} else {
  return std::forward<F>(f)();
}
\end{codeblock}
\end{itemdescr}

\indexlibrarymember{or_else}{optional}
\begin{itemdecl}
template<class F> constexpr optional or_else(F&& f) &&;
\end{itemdecl}

\begin{itemdescr}
\pnum
\constraints
\tcode{F} models \tcode{\libconcept{invocable}<>} and
\tcode{T} models \libconcept{move_constructible}.

\pnum
\mandates
\tcode{is_same_v<remove_cvref_t<invoke_result_t<F>>, optional>} is \tcode{true}.

\pnum
\effects
Equivalent to:
\begin{codeblock}
if (*this) {
  return std::move(*this);
} else {
  return std::forward<F>(f)();
}
\end{codeblock}
\end{itemdescr}

\rSec3[optional.mod]{Modifiers}

\indexlibrarymember{reset}{optional}%
\begin{itemdecl}
constexpr void reset() noexcept;
\end{itemdecl}

\begin{itemdescr}
\pnum
\effects
If \tcode{*this} contains a value, calls \tcode{val->T::\~T()} to destroy the contained value;
otherwise no effect.

\pnum
\ensures
\tcode{*this} does not contain a value.
\end{itemdescr}

\rSec2[optional.nullopt]{No-value state indicator}

\indexlibraryglobal{nullopt_t}%
\indexlibraryglobal{nullopt}%
\begin{itemdecl}
struct nullopt_t{@\seebelow@};
inline constexpr nullopt_t nullopt(@\unspec@);
\end{itemdecl}

\pnum
The struct \tcode{nullopt_t} is an empty class type used as a unique type to indicate the state of not containing a value for \tcode{optional} objects.
In particular, \tcode{optional<T>} has a constructor with \tcode{nullopt_t} as a single argument;
this indicates that an optional object not containing a value shall be constructed.

\pnum
Type \tcode{nullopt_t} shall not have a default constructor or an initializer-list constructor, and shall not be an aggregate.

\rSec2[optional.bad.access]{Class \tcode{bad_optional_access}}

\begin{codeblock}
namespace std {
  class bad_optional_access : public exception {
  public:
    // see \ref{exception} for the specification of the special member functions
    const char* what() const noexcept override;
  };
}
\end{codeblock}

\pnum
The class \tcode{bad_optional_access} defines the type of objects thrown as exceptions to report the situation where an attempt is made to access the value of an optional object that does not contain a value.

\indexlibrarymember{what}{bad_optional_access}%
\begin{itemdecl}
const char* what() const noexcept override;
\end{itemdecl}

\begin{itemdescr}
\pnum
\returns
An \impldef{return value of \tcode{bad_optional_access::what}} \ntbs{}.
\end{itemdescr}

\rSec2[optional.relops]{Relational operators}

\indexlibrarymember{operator==}{optional}%
\begin{itemdecl}
template<class T, class U> constexpr bool operator==(const optional<T>& x, const optional<U>& y);
\end{itemdecl}

\begin{itemdescr}
\pnum
\constraints
The expression \tcode{*x == *y} is well-formed and
its result is convertible to \tcode{bool}.
\begin{note}
\tcode{T} need not be \oldconcept{EqualityComparable}.
\end{note}

\pnum
\returns
If \tcode{x.has_value() != y.has_value()}, \tcode{false}; otherwise if \tcode{x.has_value() == false}, \tcode{true}; otherwise \tcode{*x == *y}.

\pnum
\remarks
Specializations of this function template
for which \tcode{*x == *y} is a core constant expression
are constexpr functions.
\end{itemdescr}

\indexlibrarymember{operator"!=}{optional}%
\begin{itemdecl}
template<class T, class U> constexpr bool operator!=(const optional<T>& x, const optional<U>& y);
\end{itemdecl}

\begin{itemdescr}
\pnum
\constraints
The expression \tcode{*x != *y} is well-formed and
its result is convertible to \tcode{bool}.

\pnum
\returns
If \tcode{x.has_value() != y.has_value()}, \tcode{true};
otherwise, if \tcode{x.has_value() == false}, \tcode{false};
otherwise \tcode{*x != *y}.

\pnum
\remarks
Specializations of this function template
for which \tcode{*x != *y} is a core constant expression
are constexpr functions.
\end{itemdescr}

\indexlibrarymember{operator<}{optional}%
\begin{itemdecl}
template<class T, class U> constexpr bool operator<(const optional<T>& x, const optional<U>& y);
\end{itemdecl}

\begin{itemdescr}
\pnum
\constraints
\tcode{*x < *y} is well-formed
and its result is convertible to \tcode{bool}.

\pnum
\returns
If \tcode{!y}, \tcode{false};
otherwise, if \tcode{!x}, \tcode{true};
otherwise \tcode{*x < *y}.

\pnum
\remarks
Specializations of this function template
for which \tcode{*x < *y} is a core constant expression
are constexpr functions.
\end{itemdescr}

\indexlibrarymember{operator>}{optional}%
\begin{itemdecl}
template<class T, class U> constexpr bool operator>(const optional<T>& x, const optional<U>& y);
\end{itemdecl}

\begin{itemdescr}
\pnum
\constraints
The expression \tcode{*x > *y} is well-formed and
its result is convertible to \tcode{bool}.

\pnum
\returns
If \tcode{!x}, \tcode{false};
otherwise, if \tcode{!y}, \tcode{true};
otherwise \tcode{*x > *y}.

\pnum
\remarks
Specializations of this function template
for which \tcode{*x > *y} is a core constant expression
are constexpr functions.
\end{itemdescr}

\indexlibrarymember{operator<=}{optional}%
\begin{itemdecl}
template<class T, class U> constexpr bool operator<=(const optional<T>& x, const optional<U>& y);
\end{itemdecl}

\begin{itemdescr}
\pnum
\constraints
The expression \tcode{*x <= *y} is well-formed and
its result is convertible to \tcode{bool}.

\pnum
\returns
If \tcode{!x}, \tcode{true};
otherwise, if \tcode{!y}, \tcode{false};
otherwise \tcode{*x <= *y}.

\pnum
\remarks
Specializations of this function template
for which \tcode{*x <= *y} is a core constant expression
are constexpr functions.
\end{itemdescr}

\indexlibrarymember{operator>=}{optional}%
\begin{itemdecl}
template<class T, class U> constexpr bool operator>=(const optional<T>& x, const optional<U>& y);
\end{itemdecl}

\begin{itemdescr}
\pnum
\constraints
The expression \tcode{*x >= *y} is well-formed and
its result is convertible to \tcode{bool}.

\pnum
\returns
If \tcode{!y}, \tcode{true};
otherwise, if \tcode{!x}, \tcode{false};
otherwise \tcode{*x >= *y}.

\pnum
\remarks
Specializations of this function template
for which \tcode{*x >= *y} is a core constant expression
are constexpr functions.
\end{itemdescr}

\indexlibrarymember{operator<=>}{optional}%
\begin{itemdecl}
template<class T, @\libconcept{three_way_comparable_with}@<T> U>
  constexpr compare_three_way_result_t<T, U>
    operator<=>(const optional<T>& x, const optional<U>& y);
\end{itemdecl}

\begin{itemdescr}
\pnum
\returns
If \tcode{x \&\& y}, \tcode{*x <=> *y}; otherwise \tcode{x.has_value() <=> y.has_value()}.

\pnum
\remarks
Specializations of this function template
for which \tcode{*x <=> *y} is a core constant expression
are constexpr functions.
\end{itemdescr}

\rSec2[optional.nullops]{Comparison with \tcode{nullopt}}

\indexlibrarymember{operator==}{optional}%
\begin{itemdecl}
template<class T> constexpr bool operator==(const optional<T>& x, nullopt_t) noexcept;
\end{itemdecl}

\begin{itemdescr}
\pnum
\returns
\tcode{!x}.
\end{itemdescr}

\indexlibrarymember{operator<=>}{optional}%
\begin{itemdecl}
template<class T> constexpr strong_ordering operator<=>(const optional<T>& x, nullopt_t) noexcept;
\end{itemdecl}

\begin{itemdescr}
\pnum
\returns
\tcode{x.has_value() <=> false}.
\end{itemdescr}

\rSec2[optional.comp.with.t]{Comparison with \tcode{T}}

\indexlibrarymember{operator==}{optional}%
\begin{itemdecl}
template<class T, class U> constexpr bool operator==(const optional<T>& x, const U& v);
\end{itemdecl}

\begin{itemdescr}
\pnum
\constraints
The expression \tcode{*x == v} is well-formed and
its result is convertible to \tcode{bool}.
\begin{note}
\tcode{T} need not be \oldconcept{EqualityComparable}.
\end{note}

\pnum
\effects
Equivalent to: \tcode{return x.has_value() ? *x == v : false;}
\end{itemdescr}

\indexlibrarymember{operator==}{optional}%
\begin{itemdecl}
template<class T, class U> constexpr bool operator==(const T& v, const optional<U>& x);
\end{itemdecl}

\begin{itemdescr}
\pnum
\constraints
The expression \tcode{v == *x} is well-formed and
its result is convertible to \tcode{bool}.

\pnum
\effects
Equivalent to: \tcode{return x.has_value() ? v == *x : false;}
\end{itemdescr}

\indexlibrarymember{operator"!=}{optional}%
\begin{itemdecl}
template<class T, class U> constexpr bool operator!=(const optional<T>& x, const U& v);
\end{itemdecl}

\begin{itemdescr}
\pnum
\constraints
The expression \tcode{*x != v} is well-formed and
its result is convertible to \tcode{bool}.

\pnum
\effects
Equivalent to: \tcode{return x.has_value() ? *x != v : true;}
\end{itemdescr}

\indexlibrarymember{operator"!=}{optional}%
\begin{itemdecl}
template<class T, class U> constexpr bool operator!=(const T& v, const optional<U>& x);
\end{itemdecl}

\begin{itemdescr}
\pnum
\constraints
The expression \tcode{v != *x} is well-formed and
its result is convertible to \tcode{bool}.

\pnum
\effects
Equivalent to: \tcode{return x.has_value() ? v != *x : true;}
\end{itemdescr}

\indexlibrarymember{operator<}{optional}%
\begin{itemdecl}
template<class T, class U> constexpr bool operator<(const optional<T>& x, const U& v);
\end{itemdecl}

\begin{itemdescr}
\pnum
\constraints
The expression \tcode{*x < v} is well-formed and
its result is convertible to \tcode{bool}.

\pnum
\effects
Equivalent to: \tcode{return x.has_value() ? *x < v : true;}
\end{itemdescr}

\indexlibrarymember{operator<}{optional}%
\begin{itemdecl}
template<class T, class U> constexpr bool operator<(const T& v, const optional<U>& x);
\end{itemdecl}

\begin{itemdescr}
\pnum
\constraints
The expression \tcode{v < *x} is well-formed and
its result is convertible to \tcode{bool}.

\pnum
\effects
Equivalent to: \tcode{return x.has_value() ? v < *x : false;}
\end{itemdescr}

\indexlibrarymember{operator>}{optional}%
\begin{itemdecl}
template<class T, class U> constexpr bool operator>(const optional<T>& x, const U& v);
\end{itemdecl}

\begin{itemdescr}
\pnum
\constraints
The expression \tcode{*x > v} is well-formed and
its result is convertible to \tcode{bool}.

\pnum
\effects
Equivalent to: \tcode{return x.has_value() ? *x > v : false;}
\end{itemdescr}

\indexlibrarymember{operator>}{optional}%
\begin{itemdecl}
template<class T, class U> constexpr bool operator>(const T& v, const optional<U>& x);
\end{itemdecl}

\begin{itemdescr}
\pnum
\constraints
The expression \tcode{v > *x} is well-formed and
its result is convertible to \tcode{bool}.

\pnum
\effects
Equivalent to: \tcode{return x.has_value() ? v > *x : true;}
\end{itemdescr}

\indexlibrarymember{operator<=}{optional}%
\begin{itemdecl}
template<class T, class U> constexpr bool operator<=(const optional<T>& x, const U& v);
\end{itemdecl}

\begin{itemdescr}
\pnum
\constraints
The expression \tcode{*x <= v} is well-formed and
its result is convertible to \tcode{bool}.

\pnum
\effects
Equivalent to: \tcode{return x.has_value() ? *x <= v : true;}
\end{itemdescr}

\indexlibrarymember{operator<=}{optional}%
\begin{itemdecl}
template<class T, class U> constexpr bool operator<=(const T& v, const optional<U>& x);
\end{itemdecl}

\begin{itemdescr}
\pnum
\constraints
The expression \tcode{v <= *x} is well-formed and
its result is convertible to \tcode{bool}.

\pnum
\effects
Equivalent to: \tcode{return x.has_value() ? v <= *x : false;}
\end{itemdescr}

\indexlibrarymember{operator>=}{optional}%
\begin{itemdecl}
template<class T, class U> constexpr bool operator>=(const optional<T>& x, const U& v);
\end{itemdecl}

\begin{itemdescr}
\pnum
\constraints
The expression \tcode{*x >= v} is well-formed and
its result is convertible to \tcode{bool}.

\pnum
\effects
Equivalent to: \tcode{return x.has_value() ? *x >= v : false;}
\end{itemdescr}

\indexlibrarymember{operator>=}{optional}%
\begin{itemdecl}
template<class T, class U> constexpr bool operator>=(const T& v, const optional<U>& x);
\end{itemdecl}

\begin{itemdescr}
\pnum
\constraints
The expression \tcode{v >= *x} is well-formed and
its result is convertible to \tcode{bool}.

\pnum
\effects
Equivalent to: \tcode{return x.has_value() ? v >= *x : true;}
\end{itemdescr}

\indexlibrarymember{operator<=>}{optional}%
\begin{itemdecl}
template<class T, class U>
    requires (!@\exposconcept{is-derived-from-optional}@<U>) && @\libconcept{three_way_comparable_with}@<T, U>
  constexpr compare_three_way_result_t<T, U>
    operator<=>(const optional<T>& x, const U& v);
\end{itemdecl}

\begin{itemdescr}
\pnum
\effects
Equivalent to: \tcode{return x.has_value() ? *x <=> v : strong_ordering::less;}
\end{itemdescr}

\rSec2[optional.specalg]{Specialized algorithms}

\indexlibrarymember{swap}{optional}%
\begin{itemdecl}
template<class T>
  constexpr void swap(optional<T>& x, optional<T>& y) noexcept(noexcept(x.swap(y)));
\end{itemdecl}

\begin{itemdescr}
\pnum
\constraints
\tcode{is_move_constructible_v<T>} is \tcode{true} and
\tcode{is_swappable_v<T>} is \tcode{true}.

\pnum
\effects
Calls \tcode{x.swap(y)}.
\end{itemdescr}

\indexlibraryglobal{make_optional}%
\begin{itemdecl}
template<class T> constexpr optional<decay_t<T>> make_optional(T&& v);
\end{itemdecl}

\begin{itemdescr}
\pnum
\returns
\tcode{optional<decay_t<T>>(std::forward<T>(v))}.
\end{itemdescr}

\indexlibraryglobal{make_optional}%
\begin{itemdecl}
template<class T, class...Args>
  constexpr optional<T> make_optional(Args&&... args);
\end{itemdecl}

\begin{itemdescr}
\pnum
\effects
Equivalent to: \tcode{return optional<T>(in_place, std::forward<Args>(args)...);}
\end{itemdescr}

\indexlibraryglobal{make_optional}%
\begin{itemdecl}
template<class T, class U, class... Args>
  constexpr optional<T> make_optional(initializer_list<U> il, Args&&... args);
\end{itemdecl}

\begin{itemdescr}
\pnum
\effects
Equivalent to: \tcode{return optional<T>(in_place, il, std::forward<Args>(args)...);}
\end{itemdescr}

\rSec2[optional.hash]{Hash support}

\indexlibrarymember{hash}{optional}%
\begin{itemdecl}
template<class T> struct hash<optional<T>>;
\end{itemdecl}

\begin{itemdescr}
\pnum
The specialization \tcode{hash<optional<T>>} is enabled\iref{unord.hash}
if and only if \tcode{hash<remove_const_t<T>>} is enabled.
When enabled, for an object \tcode{o} of type \tcode{optional<T>},
if \tcode{o.has_value() == true}, then \tcode{hash<optional<T>>()(o)}
evaluates to the same value as \tcode{hash<remove_const_t<T>>()(*o)};
otherwise it evaluates to an unspecified value.
The member functions are not guaranteed to be \keyword{noexcept}.
\end{itemdescr}


\end{wording}

\chapter{Impact on the standard}

A pure library extension, affecting no other parts of the library or language.

The proposed changes are relative to the current working draft \cite{N4910}.

\chapter*{Document history}

\begin{itemize}
\item \textbf{Changes since R1}
  \begin{itemize}
  \item Design points called out
  \end{itemize}
\item \textbf{Changes since R0}
  \begin{itemize}
  \item Wording Updates
  \end{itemize}
\end{itemize}

\renewcommand{\bibname}{References}
\bibliographystyle{abstract}
\bibliography{wg21,mybiblio}


\backmatter
\chapter*{Implementation}

\begin{minted}{c++}
  // ----------------------
  // BASE AND DETAILS ELIDED
  // ----------------------

  /****************/
  /* optional<T&> */
  /****************/

  template <class T>
  class optional<T&> {
    public:
    using value_type = T&;
    using iterator =
    detail::contiguous_iterator<T,
    optional>; // see [optionalref.iterators]
    using const_iterator =
    detail::contiguous_iterator<const T,
    optional>; // see [optionalref.iterators]

    private:
    template <class R, class Arg>
    constexpr R make_reference(Arg&& arg) // exposition only
    requires is_constructible_v<R, Arg>;

    public:
    // \ref{optionalref.ctor}, constructors

    constexpr optional() noexcept;
    constexpr optional(nullopt_t) noexcept;
    constexpr optional(const optional& rhs) noexcept = default;
    constexpr optional(optional&& rhs) noexcept      = default;

    template <class Arg>
    constexpr explicit optional(in_place_t, Arg&& arg)
    requires is_constructible_v<add_lvalue_reference_t<T>, Arg>;

    template <class U = T>
    requires(!detail::is_optional<decay_t<U>>)
    constexpr explicit(!is_convertible_v<U, T>) optional(U&& u) noexcept;
    template <class U>
    constexpr explicit(!is_convertible_v<U, T>)
    optional(const optional<U>& rhs) noexcept;

    // \ref{optionalref.dtor}, destructor
    constexpr ~optional() = default;

    // \ref{optionalref.assign}, assignment
    constexpr optional& operator=(nullopt_t) noexcept;

    constexpr optional& operator=(const optional& rhs) noexcept = default;
    constexpr optional& operator=(optional&& rhs) noexcept      = default;

    template <class U = T>
    requires(!detail::is_optional<decay_t<U>>)
    constexpr optional& operator=(U&& u);

    template <class U>
    constexpr optional& operator=(const optional<U>& rhs) noexcept;

    template <class U>
    constexpr optional& operator=(optional<U>&& rhs);

    template <class U>
    requires(!detail::is_optional<decay_t<U>>)
    constexpr optional& emplace(U&& u) noexcept;

    // \ref{optionalref.swap}, swap
    constexpr void swap(optional& rhs) noexcept;

    // \ref{optional.iterators}, iterator support
    constexpr iterator       begin() noexcept;
    constexpr const_iterator begin() const noexcept;
    constexpr iterator       end() noexcept;
    constexpr const_iterator end() const noexcept;

    // \ref{optionalref.observe}, observers
    constexpr T*       operator->() const noexcept;
    constexpr T&       operator*() const noexcept;
    constexpr explicit operator bool() const noexcept;
    constexpr bool     has_value() const noexcept;
    constexpr T&       value() const;
    template <class U>
    constexpr T value_or(U&& u) const;

    // \ref{optionalref.monadic}, monadic operations
    template <class F>
    constexpr auto and_then(F&& f) const;
    template <class F>
    constexpr auto transform(F&& f) const -> optional<invoke_result_t<F, T&>>;
    template <class F>
    constexpr optional or_else(F&& f) const;

    // \ref{optional.mod}, modifiers
    constexpr void reset() noexcept;

    private:
    T* value_; // exposition only
  };

  template <class T>
  template <class R, class Arg>
  constexpr R optional<T&>::make_reference(Arg&& arg)
  requires is_constructible_v<R, Arg>
  {
    static_assert(std::is_reference_v<R>);
    #if (__cpp_lib_reference_from_temporary >= 202202L)
    static_assert(!std::reference_converts_from_temporary_v<R, Arg>,
    "Reference conversion from temporary not allowed.");
    #endif
    R r(std::forward<Arg>(arg));
    return r;
  }

  //  \rSec3[optionalref.ctor]{Constructors}
  template <class T>
  constexpr optional<T&>::optional() noexcept : value_(nullptr) {}

  template <class T>
  constexpr optional<T&>::optional(nullopt_t) noexcept : value_(nullptr) {}

  template <class T>
  template <class Arg>
  constexpr optional<T&>::optional(in_place_t, Arg&& arg)
  requires is_constructible_v<add_lvalue_reference_t<T>, Arg>
  : value_(addressof(
  make_reference<add_lvalue_reference_t<T>>(std::forward<Arg>(arg)))) {
  }

  template <class T>
  template <class U>
  requires(!detail::is_optional<decay_t<U>>)
  constexpr optional<T&>::optional(U&& u) noexcept : value_(addressof(u)) {
    static_assert(is_constructible_v<add_lvalue_reference_t<T>, U>,
    "Must be able to bind U to T&");
    static_assert(is_lvalue_reference<U>::value, "U must be an lvalue");
  }

  template <class T>
  template <class U>
  constexpr optional<T&>::optional(const optional<U>& rhs) noexcept {
    static_assert(is_constructible_v<add_lvalue_reference_t<T>, U>,
    "Must be able to bind U to T&");
    if (rhs.has_value())
    value_ = to_address(rhs);
    else
    value_ = nullptr;
  }

  // \rSec3[optionalref.assign]{Assignment}
  template <class T>
  constexpr optional<T&>& optional<T&>::operator=(nullopt_t) noexcept {
    value_ = nullptr;
    return *this;
  }

  template <class T>
  template <class U>
  requires(!detail::is_optional<decay_t<U>>)
  constexpr optional<T&>& optional<T&>::operator=(U&& u) {
    static_assert(is_constructible_v<add_lvalue_reference_t<T>, U>,
    "Must be able to bind U to T&");
    static_assert(is_lvalue_reference<U>::value, "U must be an lvalue");
    value_ = addressof(u);
    return *this;
  }

  template <class T>
  template <class U>
  constexpr optional<T&>&
  optional<T&>::operator=(const optional<U>& rhs) noexcept {
    static_assert(is_constructible_v<add_lvalue_reference_t<T>, U>,
    "Must be able to bind U to T&");
    if (rhs.has_value())
    value_ = to_address(rhs);
    else
    value_ = nullptr;
    return *this;
  }

  template <class T>
  template <class U>
  constexpr optional<T&>& optional<T&>::operator=(optional<U>&& rhs) {
    static_assert(is_constructible_v<add_lvalue_reference_t<T>, U>,
    "Must be able to bind U to T&");
    #if (__cpp_lib_reference_from_temporary >= 202202L)
    static_assert(
    !std::reference_converts_from_temporary_v<add_lvalue_reference_t<T>,
    U>,
    "Reference conversion from temporary not allowed.");
    #endif
    if (rhs.has_value())
    value_ = to_address(rhs);
    else
    value_ = nullptr;
    return *this;
  }

  template <class T>
  template <class U>
  requires(!detail::is_optional<decay_t<U>>)
  constexpr optional<T&>& optional<T&>::emplace(U&& u) noexcept {
    return *this = std::forward<U>(u);
  }

  //   \rSec3[optionalref.swap]{Swap}

  template <class T>
  constexpr void optional<T&>::swap(optional<T&>& rhs) noexcept {
    std::swap(value_, rhs.value_);
  }

  // \rSec3[optionalref.iterators]{Iterator Support}
  template <class T>
  constexpr optional<T&>::iterator optional<T&>::begin() noexcept {
    return iterator(has_value() ? value_ : nullptr);
  };

  template <class T>
  constexpr optional<T&>::const_iterator optional<T&>::begin() const noexcept {
    return const_iterator(has_value() ? value_ : nullptr);
  };

  template <class T>
  constexpr optional<T&>::iterator optional<T&>::end() noexcept {
    return begin() + has_value();
  }

  template <class T>
  constexpr optional<T&>::const_iterator optional<T&>::end() const noexcept {
    return begin() + has_value();
  }

  // \rSec3[optionalref.observe]{Observers}
  template <class T>
  constexpr T* optional<T&>::operator->() const noexcept {
    return value_;
  }

  template <class T>
  constexpr T& optional<T&>::operator*() const noexcept {
    return *value_;
  }

  template <class T>
  constexpr optional<T&>::operator bool() const noexcept {
    return value_ != nullptr;
  }
  template <class T>
  constexpr bool optional<T&>::has_value() const noexcept {
    return value_ != nullptr;
  }

  template <class T>
  constexpr T& optional<T&>::value() const {
    if (has_value())
    return *value_;
    throw bad_optional_access();
  }

  template <class T>
  template <class U>
  constexpr T optional<T&>::value_or(U&& u) const {
    static_assert(is_copy_constructible_v<T>, "T must be copy constructible");
    static_assert(is_convertible_v<decltype(u), T>,
    "Must be able to convert u to T");
    return has_value() ? *value_ : std::forward<U>(u);
  }

  //   \rSec3[optionalref.monadic]{Monadic operations}
  template <class T>
  template <class F>
  constexpr auto optional<T&>::and_then(F&& f) const {
    using U = invoke_result_t<F, T&>;
    static_assert(detail::is_optional<U>, "F must return an optional");
    return (has_value()) ? invoke(std::forward<F>(f), *value_)
    : remove_cvref_t<U>();
  }

  template <class T>
  template <class F>
  constexpr auto optional<T&>::transform(F&& f) const
  -> optional<invoke_result_t<F, T&>> {
    using U = invoke_result_t<F, T&>;
    return (has_value()) ? optional<U>{invoke(std::forward<F>(f), *value_)}
    : optional<U>{};
  }

  template <class T>
  template <class F>
  constexpr optional<T&> optional<T&>::or_else(F&& f) const {
    using U = invoke_result_t<F>;
    static_assert(is_same_v<remove_cvref_t<U>, optional>);
    return has_value() ? *value_ : std::forward<F>(f)();
  }

  // \rSec3[optional.mod]{modifiers}
  template <class T>
  constexpr void optional<T&>::reset() noexcept {
    value_ = nullptr;
  }

\end{minted}
\end{document}
